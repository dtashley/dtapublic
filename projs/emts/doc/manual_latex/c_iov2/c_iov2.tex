%%%%%%%%%%%%%%%%%%%%%%%%%%%%%%%%%%%%%%%%%%%%%%%%%%%%%%%%%%%%%%%%%%%%%%%%%%%%%%%%%%%%%%%%%%%%%%%%%%%%
\chapter{Introduction and Overview}

\label{ciov0}

%%%%%%%%%%%%%%%%%%%%%%%%%%%%%%%%%%%%%%%%%%%%%%%%%%%%%%%%%%%%%%%%%%%%%%%%%%%%%%%%%%%%%%%%%%%%%%%%%%%%
%%%%%%%%%%%%%%%%%%%%%%%%%%%%%%%%%%%%%%%%%%%%%%%%%%%%%%%%%%%%%%%%%%%%%%%%%%%%%%%%%%%%%%%%%%%%%%%%%%%%
%%%%%%%%%%%%%%%%%%%%%%%%%%%%%%%%%%%%%%%%%%%%%%%%%%%%%%%%%%%%%%%%%%%%%%%%%%%%%%%%%%%%%%%%%%%%%%%%%%%%
\section{General Description}
\label{ciov2:siov0}

The Embedded Tool Set (or EMTS) is a set of open-source software tools
to support the development of embedded systems.  These software tools run
on:

\begin{enumerate}
   \item \emph{Windows}-based personal computers.
   \item \emph{*nix}-based personal computers.
   \item \emph{*nix} servers.
   \item \emph{Android}-based phones/tablets.
   \item \emph{iOS}-based phones/tablets.
   \item \emph{FireOS}-based phones/tablets.
   \item Embedded systems.
\end{enumerate}


%%%%%%%%%%%%%%%%%%%%%%%%%%%%%%%%%%%%%%%%%%%%%%%%%%%%%%%%%%%%%%%%%%%%%%%%%%%%%%%%%%%%%%%%%%%%%%%%%%%%
%%%%%%%%%%%%%%%%%%%%%%%%%%%%%%%%%%%%%%%%%%%%%%%%%%%%%%%%%%%%%%%%%%%%%%%%%%%%%%%%%%%%%%%%%%%%%%%%%%%%
%%%%%%%%%%%%%%%%%%%%%%%%%%%%%%%%%%%%%%%%%%%%%%%%%%%%%%%%%%%%%%%%%%%%%%%%%%%%%%%%%%%%%%%%%%%%%%%%%%%%
\section{Taxonomy and Naming Conventions}
\label{ciov2:stnc0}

\begin{itemize}
   \item Tool Set (TS)
      \begin{itemize}
         \item The entire EMTS tool set.
      \end{itemize}
   \item Tool Collection (TC)
      \begin{itemize}
         \item A set of logical tools that are packaged together into one executable.
      \end{itemize}
   \item Tool (TL)
      \begin{itemize}
         \item Cohesive functionality that can never be split for
               the purposes of packaging.
            \begin{itemize}
               \item Any TS component component either includes
                     the TL in its entirety or not at all.
            \end{itemize}
         \item Makes its own threading decisions.
            \begin{itemize}
               \item In most applications of the TL, it is provided with
                     threads to use (mechanism TBD).  (It may not create its own.)
               \item It may use the provided threads as it pleases.
               \item It must relinquish the threads on loss of focus or other
                     similar event (mechanism TBD).
            \end{itemize}
      \end{itemize}
   \item Tool Executable (TLE)
      \begin{itemize}
         \item An executable containing only a single tool.
      \end{itemize}
   \item Tool Script (TSC)
      \begin{itemize}
         \item Script component(s) implementing tool functionality.
         \item Implemented in PHP, Python, Perl, etc.
      \end{itemize}
   \item Tool Web Interface (TWI)
      \begin{itemize}
         \item Script component(s) that generate the web interface to tool functionality.
         \item Implemented in PHP, Python, Perl, etc.
      \end{itemize}
   \item Tool Function (TF)
      \begin{itemize}
         \item An individual function within a tool.
      \end{itemize}
   \item Tool Function Executable (TFE)
      \begin{itemize}
         \item An executable containing only an individual tool function.
      \end{itemize}
   \item Tool Function Script (TFSC)
      \begin{itemize}
         \item Script component(s) implementing only an individual tool function.
         \item Implemented in PHP, Python, Perl, etc.
      \end{itemize}
   \item Tool Function Web Interface (TFWI)
      \begin{itemize}
         \item Script component(s) that generate the web interface to a tool function.
         \item Implemented in PHP, Python, Perl, etc.
      \end{itemize}
\end{itemize}

Tools and tool functions are packaged within the tool set in 4 ways:

\begin{itemize}
   \item As built-in functions for a script interpreter.\footnote{The script interpreter
         interprets a language called Clike, described fully in \S{}TBD.  The interpreter
         can be used both interactively (in this mode, it might be described as a
         \emph{very} powerful calculator) and to run scripts (which give programmatic access
         to all EMTS functionality).}
   \item As GUI interfaces within executables.
   \item As stand-alone console-mode executables that implement specific sets of tools or
         tool functions.
   \item As functionality implemented on a web page.
\end{itemize}

%%%%%%%%%%%%%%%%%%%%%%%%%%%%%%%%%%%%%%%%%%%%%%%%%%%%%%%%%%%%%%%%%%%%%%%%%%%%%%%%%%%%%%%%%%%%%%%%%%%%
%%%%%%%%%%%%%%%%%%%%%%%%%%%%%%%%%%%%%%%%%%%%%%%%%%%%%%%%%%%%%%%%%%%%%%%%%%%%%%%%%%%%%%%%%%%%%%%%%%%%
%%%%%%%%%%%%%%%%%%%%%%%%%%%%%%%%%%%%%%%%%%%%%%%%%%%%%%%%%%%%%%%%%%%%%%%%%%%%%%%%%%%%%%%%%%%%%%%%%%%%
\section{License}\index{license}
\label{ciov2:slic0}

This document, all computer and paper files and records used to create and
distribute this document, the software described by this document, and all computer
and paper files and records used to create and distribute the software described by
this document, are provided under \index{MIT license@\emph{The MIT License}}\emph{The MIT License},
reproduced immediately below.\\

\begin{small}
\noindent{}\emph{Copyright \copyright 2020 David T. Ashley}\\\\
\noindent{}\emph{Permission is hereby granted, free of charge, to any person obtaining a copy
of this software source code and associated documentation files (the
``Software''), to deal in the Software without restriction, including without
limitation the rights to use, copy, modify, merge, publish, distribute,
sublicense, and/or sell copies of the Software, and to permit persons to whom
the Software is furnished to do so, subject to the following conditions:}

\begin{itemize}
\item \emph{The above copyright notice and this permission notice shall be included in
      all copies or substantial portions of the Software.}
\item \emph{THE SOFTWARE IS PROVIDED ``AS IS'', WITHOUT WARRANTY OF ANY KIND, EXPRESS OR
      IMPLIED, INCLUDING BUT NOT LIMITED TO THE WARRANTIES OF MERCHANTABILITY,
      FITNESS FOR A PARTICULAR PURPOSE AND NONINFRINGEMENT\@. IN NO EVENT SHALL THE
      AUTHORS OR COPYRIGHT HOLDERS BE LIABLE FOR ANY CLAIM, DAMAGES OR OTHER
      LIABILITY, WHETHER IN AN ACTION OF CONTRACT, TORT OR OTHERWISE, ARISING FROM,
      OUT OF OR IN CONNECTION WITH THE SOFTWARE OR THE USE OR OTHER DEALINGS IN
      THE SOFTWARE.}
\end{itemize}
\end{small}

A substantial benefit of \emph{The MIT License} is that it does not require manufacturers
of embedded system products to provide notice to customers that open-source
software is incorporated into the product.\footnote{At least that is \emph{my} interpretation
of the license.  The \emph{Open Source Initiative} website suggests that \emph{software} means
the source code, and that binaries are a separate matter.  I'm willing to discuss
this interpretation and to move to an even less restrictive license if there are
concerns about a manufacturer's obligations.}  A typical consumer has no interest in the
origin of the software in a product he purchases.  A typical manufacturer does not want
the burden of providing notice.

%%%%%%%%%%%%%%%%%%%%%%%%%%%%%%%%%%%%%%%%%%%%%%%%%%%%%%%%%%%%%%%%%%%%%%%%%%%%%%%%%%%%%%%%%%%%%%%%%%%%
%%%%%%%%%%%%%%%%%%%%%%%%%%%%%%%%%%%%%%%%%%%%%%%%%%%%%%%%%%%%%%%%%%%%%%%%%%%%%%%%%%%%%%%%%%%%%%%%%%%%
%%%%%%%%%%%%%%%%%%%%%%%%%%%%%%%%%%%%%%%%%%%%%%%%%%%%%%%%%%%%%%%%%%%%%%%%%%%%%%%%%%%%%%%%%%%%%%%%%%%%
\section{Companion Book}\index{companion book}
\label{ciov2:scbk0}

TBD.

%%%%%%%%%%%%%%%%%%%%%%%%%%%%%%%%%%%%%%%%%%%%%%%%%%%%%%%%%%%%%%%%%%%%%%%%%%%%%%%%%%%%%%%%%%%%%%%%%%%%
%%%%%%%%%%%%%%%%%%%%%%%%%%%%%%%%%%%%%%%%%%%%%%%%%%%%%%%%%%%%%%%%%%%%%%%%%%%%%%%%%%%%%%%%%%%%%%%%%%%%
%%%%%%%%%%%%%%%%%%%%%%%%%%%%%%%%%%%%%%%%%%%%%%%%%%%%%%%%%%%%%%%%%%%%%%%%%%%%%%%%%%%%%%%%%%%%%%%%%%%%
\section{Choice of \emph{\LaTeX{}}}
\label{ciov2:sclt0}

\LaTeX{} is an uncommon word processing tool.\footnote{The distribution I use is
\index{MikTeX@\emph{MikTeX}}\emph{MikTeX}.}  Its key features are:

\begin{itemize}
   \item Text source files are prepared, and then compiled to graphical output (in this respect,
         it resembles a programming language compiler).
   \item Files containing figures are usually in individual files, separate from the source
         files.  The source files reference the files containing the figures.
   \item The references that have to be resolved as part of compilation (table of contents
         entries, index entries, cross-references) are stored in auxiliary text files that are
         generated and
         used by the compilation process.  (Simple tools can be written
         to use this information.)
\end{itemize}

\LaTeX{} was chosen for this document for these reasons:

\begin{itemize}
   \item \LaTeX{} source files (plain text) are very compatible with \emph{git}.
         Difference reports are meaning and can be understood for review.
   \item Figures are stored as separate \index{eps file@\emph{.eps} file}\emph{.eps} files.
         These change infrequently and allow \emph{git} to deal with typical document changes
         (changes to plain text files only) efficiently.  (This is unlike a typical
         \index{Microsoft Word@\emph{Microsoft Word}}\emph{Microsoft Word} document, where
         \emph{git} is forced to treat the document as binary and store a new copy of the entire
         document when any minor change is made.)
   \item By processing the auxiliary text files that are produced during 
         \LaTeX{} compilation, this document and the companion book may contain section
         and page references to each other.
\end{itemize}

%%%%%%%%%%%%%%%%%%%%%%%%%%%%%%%%%%%%%%%%%%%%%%%%%%%%%%%%%%%%%%%%%%%%%%%%%%%%%%%%%%%%%%%%%%%%%%%%%%%%
%%%%%%%%%%%%%%%%%%%%%%%%%%%%%%%%%%%%%%%%%%%%%%%%%%%%%%%%%%%%%%%%%%%%%%%%%%%%%%%%%%%%%%%%%%%%%%%%%%%%
%%%%%%%%%%%%%%%%%%%%%%%%%%%%%%%%%%%%%%%%%%%%%%%%%%%%%%%%%%%%%%%%%%%%%%%%%%%%%%%%%%%%%%%%%%%%%%%%%%%%
\section{Organization and Taxonomy of \emph{EMTS}}
\label{ciov2:sote0}

TBD.


%%%%%%%%%%%%%%%%%%%%%%%%%%%%%%%%%%%%%%%%%%%%%%%%%%%%%%%%%%%%%%%%%%%%%%%%%%%%%%%%%%%%%%%%%%%%%%%%%%%%
%%%%%%%%%%%%%%%%%%%%%%%%%%%%%%%%%%%%%%%%%%%%%%%%%%%%%%%%%%%%%%%%%%%%%%%%%%%%%%%%%%%%%%%%%%%%%%%%%%%%
%%%%%%%%%%%%%%%%%%%%%%%%%%%%%%%%%%%%%%%%%%%%%%%%%%%%%%%%%%%%%%%%%%%%%%%%%%%%%%%%%%%%%%%%%%%%%%%%%%%%
\subsection{Tools, Tool Collections, Tool Aggregations, and Components}
\label{ciov2:sote0:snom0}

As used in this document:

\begin{itemize}
   \item A \index{tool!definition}\emph{tool} is a software component
         or software program that performs a task or a set of closely related tasks, and is
         typically packaged as one executable or one script.
         (For example, a tool that performs arithmetic on large integers might add, subtract, multiply,
         divide large integers---tasks which are closely related.)
   \item A \index{tool collection!definition}\emph{tool collection} is a set
         of tools that are separate scripts or executables, but distributed together.
   \item A \index{tool aggregation!definition}\emph{tool aggregation} is a single executable or
         script that incorporates two or more tools, and where the functionality of the
         incorporated tools is identified distinctly.
   \item A \index{component!defined}\emph{component} is a software building block that
         is used in the construction of other software. 
\end{itemize}



%%%%%%%%%%%%%%%%%%%%%%%%%%%%%%%%%%%%%%%%%%%%%%%%%%%%%%%%%%%%%%%%%%%%%%%%%%%%%%%%%%%%%%%%%%%%%%%%%%%%
%%%%%%%%%%%%%%%%%%%%%%%%%%%%%%%%%%%%%%%%%%%%%%%%%%%%%%%%%%%%%%%%%%%%%%%%%%%%%%%%%%%%%%%%%%%%%%%%%%%%
%%%%%%%%%%%%%%%%%%%%%%%%%%%%%%%%%%%%%%%%%%%%%%%%%%%%%%%%%%%%%%%%%%%%%%%%%%%%%%%%%%%%%%%%%%%%%%%%%%%%
\subsection{Supported Platforms}
\label{ciov2:sote0:ssup0}

TBD.

%%%%%%%%%%%%%%%%%%%%%%%%%%%%%%%%%%%%%%%%%%%%%%%%%%%%%%%%%%%%%%%%%%%%%%%%%%%%%%%%%%%%%%%%%%%%%%%%%%%%
%%%%%%%%%%%%%%%%%%%%%%%%%%%%%%%%%%%%%%%%%%%%%%%%%%%%%%%%%%%%%%%%%%%%%%%%%%%%%%%%%%%%%%%%%%%%%%%%%%%%
%%%%%%%%%%%%%%%%%%%%%%%%%%%%%%%%%%%%%%%%%%%%%%%%%%%%%%%%%%%%%%%%%%%%%%%%%%%%%%%%%%%%%%%%%%%%%%%%%%%%
\section{Acknowledgements}
\label{ciov2:sack0}

TBD.

%%%%%%%%%%%%%%%%%%%%%%%%%%%%%%%%%%%%%%%%%%%%%%%%%%%%%%%%%%%%%%%%%%%%%%%%%%%%%%%%%%%%%%%%%%%%%%%%%%%%

