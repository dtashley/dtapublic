\chapter{\cquazerolongtitle{}}

\label{cqua0}

\section{Introduction}
%Section tag INT

A microcontroller can inherently manipulate only integers:  usually 8-bit
integers, 16-bit integers, or 32-bit integers.  Typically, the less
expensive a microcontroller is, the smaller the maximum data sizes that
it can accomodate; the least expensive devices can easily manipulate
only integers no larger than 8 bits.

This chapter deals with the error analysis of \emph{quantization}.  By
\emph{quantization}, we mean three [distinct] mechanisms of error introduction
in microcontroller software.

\begin{itemize}
\item \textbf{Input Quantization:} the unavoidable conversion from
      $\vworkrealset$ to $\vworkintset$ performed by interface hardware.
      For example, interface hardware may convert a voltage which is
      conceptually continuous to an integer (which can assume only discrete
      values), or may convert a time period which is conceptually
      continuous to an integer.
\item \textbf{Arithmetic Quantization:} microcontroller arithmetic algorithms
      must often discard precision in order to restrict intermediate results of
      calculations to data sizes which the microcontroller can economically
      manipulate.  This type of quantization error is most often injected
      by discarding the remainder of an integer quotient.
\item \textbf{Output Quantization:} microcontroller output hardware can
      produce continuous outputs (such as voltages or pulse widths)
      only in discrete steps.  Often, this lack of ability to control
      outputs precisely introduces additional uncertainty into the
      system which must be analyzed.
\end{itemize}

Note that the error-injection mechanism that we call \emph{quantization}
(in some sense, the creation of a discrete-\emph{data} system)
is not related to and is orthogonal to the notion of a
discrete-\emph{time} (or sampled) system.


\section{Modeling Of Quantization}
%Section tag: MOQ

For the analytical treatment of quantization, the \emph{floor($\cdot{}$)}
function, denoted $\lfloor \cdot \rfloor$, is used, often preceded by a
scaling factor.  For example, in the case of an A/D converter which
converts a voltage $\in [0, 5]$ volts into an integer
$\in [0,255]_{\vworkintsetnonneg{}}$, we may model the function which
maps from voltage to A/D count as

\begin{equation}
\label{eq:cqua0:smoq:001}
f(x) = \left\lfloor {\frac{255 x }{5}} \right\rfloor .
\end{equation}

Inherent in (\ref{eq:cqua0:smoq:001})
is the assumption that quantization will
choose an integer by rounding \emph{down}.  Other
assumptions are possible
(\ref{eq:cqua0:smoq:002}, \ref{eq:cqua0:smoq:003}).

\begin{equation}
\label{eq:cqua0:smoq:002}
f(x) = \left\lceil {\frac{255 x }{5}} \right\rceil
\end{equation}

\begin{equation}
\label{eq:cqua0:smoq:003}
f(x) = \left\lfloor {\frac{255 x }{5} + \frac{1}{2}} \right\rfloor
\end{equation}

At first glance, it may seem intuitively likely
that (\ref{eq:cqua0:smoq:003}) leads
to smaller error terms than (\ref{eq:cqua0:smoq:001})
or (\ref{eq:cqua0:smoq:002})---that rounding to the
nearest integer is a better strategy than rounding
down or rounding up.  In this case, intuition may be misleading.
(\ref{eq:cqua0:smoq:003}) more precisely \emph{centers the
expected value} of the error than (\ref{eq:cqua0:smoq:001})
or (\ref{eq:cqua0:smoq:002}), but the \emph{span} of the
error---the largest error minus the smallest error---remains
one.  In a practical system, the \emph{span} of the error
is the dominant effect.  In practice,
(\ref{eq:cqua0:smoq:001}), (\ref{eq:cqua0:smoq:002}),
and (\ref{eq:cqua0:smoq:003}) lead to near-identical
error terms.  For algebraic convenience,
(\ref{eq:cqua0:smoq:001}) is used preferentially.

Error terms are denoted by the Greek
letter \emph{epsilon} ($\varepsilon$) and
are viewed as the perturbation to the
``ideal'' to yield the ``actual''; so that a negative error
term leads to a result less than than it
``should'' be, and a positive
error term leads to a result greater than
it ``should'' be.  If the \emph{floor($\cdot{}$)}
is used to model quantization, the relationship
in (\ref{eq:cqua0:smoq:004}) holds.

\begin{equation}
\label{eq:cqua0:smoq:004}
\lfloor x \rfloor = x - \varepsilon{}; \; \varepsilon \in [0,1)
\end{equation}


\section{Error Analysis Of Addition Of Quantized Inputs}
%Section tag: eaqi

If we add two quantized values $\lfloor a \rfloor$ and
$\lfloor b \rfloor$, both $a$ and $b$ contain quantization
error, and a question of interest is how much
error the sum $\lfloor a \rfloor + \lfloor b \rfloor$ may
contain; that is, how different it may be from $a+b$.\footnote{For
addition and subtraction, this question is nearly trivial; but for
multiplication and division the relationships are more complex; and for
an arbitrary network of addition, subtraction, multiplication, and
division we are not sure how to answer this question easily.
Please see \ldots{}.}
We seek an inequality which bounds

\begin{equation}
\label{eq:cqua0:eaqi:001}
\varepsilon{} = \left( {\lfloor a \rfloor + \lfloor b \rfloor} \right)
                      - \left( {a + b} \right)  .
\end{equation}

Noting that quantization introduces an error $\varepsilon \in [0,1)$
(Eq. \ref{eq:cqua0:eaqi:001}) leads to
(\ref{eq:cqua0:eaqi:002}) and (\ref{eq:cqua0:eaqi:003}), which are equivalent statements.

\begin{equation}
\label{eq:cqua0:eaqi:002}
a + b - 2 < \lfloor a \rfloor + \lfloor b \rfloor \leq a + b
\end{equation}

\begin{equation}
\label{eq:cqua0:eaqi:003}
\varepsilon \in (-2,0]
\end{equation}

Extending (\ref{eq:cqua0:eaqi:002}) and (\ref{eq:cqua0:eaqi:003})
to an arbitrary number $N \in \vworkintsetpos{}$ of quantized inputs leads to
(\ref{eq:cqua0:eaqi:004}) and (\ref{eq:cqua0:eaqi:005}),
which are equivalent statements.

\begin{equation}
\label{eq:cqua0:eaqi:004}
\sum_{i=1}^{N} x_i - N < \sum_{i=1}^{N} \lfloor x_i \rfloor \leq \sum_{i=1}^{N} x_i
\end{equation}

\begin{equation}
\label{eq:cqua0:eaqi:005}
\varepsilon \in (-N,0]
\end{equation}

\section{Error Analysis Of Subtraction Of Quantized Inputs}

\section{Error Analysis Of Multiplication Of Quantized Inputs}

\section{Error Analysis Of Division Of Quantized Inputs}

\begin{equation}
\frac{p-1}{q}
<
\frac{\lfloor p \rfloor}{\lfloor q \rfloor}
<
\frac{p}{q-1}; \; p,q > 1
\end{equation}

\section{Error Analysis Of Arbitrary Algebraic Functions}

\section{Error Analysis Of Rational Sweeps}

\section{Exercises}

%End of file c_qua0.tex

