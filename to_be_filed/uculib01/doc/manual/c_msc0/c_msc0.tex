%$Header: /home/dashley/cvsrep/uculib01/uculib01/doc/manual/c_msc0/c_msc0.tex,v 1.11 2010/06/11 15:38:58 dashley Exp $

\chapter{Miscellaneous Functions}        
\label{cmsc0}

%%%%%%%%%%%%%%%%%%%%%%%%%%%%%%%%%%%%%%%%%%%%%%%%%%%%%%%%%%%%%%%%%%%%%%%%%%%%%%%
%%%%%%%%%%%%%%%%%%%%%%%%%%%%%%%%%%%%%%%%%%%%%%%%%%%%%%%%%%%%%%%%%%%%%%%%%%%%%%%
%%%%%%%%%%%%%%%%%%%%%%%%%%%%%%%%%%%%%%%%%%%%%%%%%%%%%%%%%%%%%%%%%%%%%%%%%%%%%%%
\section{Introduction and Overview}
%Section tag:  iov0
\label{cmsc0:siov0}

This chapter documents functions that couldn't be easily classified elsewhere.

\begin{itemize}
\item \S{}\ref{cmsc0:slfv0} (p. \pageref{cmsc0:slfv0}) documents functions that
      make available library version information. 
\end{itemize}

%%%%%%%%%%%%%%%%%%%%%%%%%%%%%%%%%%%%%%%%%%%%%%%%%%%%%%%%%%%%%%%%%%%%%%%%%%%%%%%
%%%%%%%%%%%%%%%%%%%%%%%%%%%%%%%%%%%%%%%%%%%%%%%%%%%%%%%%%%%%%%%%%%%%%%%%%%%%%%%
%%%%%%%%%%%%%%%%%%%%%%%%%%%%%%%%%%%%%%%%%%%%%%%%%%%%%%%%%%%%%%%%%%%%%%%%%%%%%%%
\section{Library Version Functions}
%Section tag:  lvf0
\label{cmsc0:slfv0}



%%%%%%%%%%%%%%%%%%%%%%%%%%%%%%%%%%%%%%%%%%%%%%%%%%%%%%%%%%%%%%%%%%%%%%%%%%%%%%%
%%%%%%%%%%%%%%%%%%%%%%%%%%%%%%%%%%%%%%%%%%%%%%%%%%%%%%%%%%%%%%%%%%%%%%%%%%%%%%%
%%%%%%%%%%%%%%%%%%%%%%%%%%%%%%%%%%%%%%%%%%%%%%%%%%%%%%%%%%%%%%%%%%%%%%%%%%%%%%%
\subsection[\emph{UcuMfLibVerMajRxx(\protect\mbox{\protect$\cdot$})}]
           {\emph{UcuMfLibVerMajRxx(\protect\mbox{\protect\boldmath $\cdot$})}}
%Section tag:  lvf0
\label{cmsc0:slvf0}

\index{UcuMfLibVerMajRxx()@\emph{UcuMfLibVerMajRxx($\cdot$)}}%

\noindent\textbf{PROTOTYPE}
\begin {list}{}{\setlength{\leftmargin}{0.25in}\setlength{\topsep}{0.0in}}
\item
\begin{verbatim}
UCU_UINT16 UcuMfLibVerMajRxx(void)
\end{verbatim}
\end{list}
\vspace{2.8ex}

\noindent\textbf{SYNOPSIS}
\begin{list}{}{\setlength{\leftmargin}{0.25in}\setlength{\topsep}{0.0in}}
\item
Returns the major version number of the library.  (For example,
if the library is version 2.5c, the integer returned would be 2.)
\end{list}
\vspace{2.8ex}

\noindent\textbf{INPUT}
\begin{list}{}{\setlength{\leftmargin}{0.5in}\setlength{\itemindent}{-0.25in}\setlength{\topsep}{0.0in}\setlength{\partopsep}{0.0in}}
\item None.
\end{list}
\vspace{2.8ex}

\noindent\textbf{OUTPUT}
\begin{list}{}{\setlength{\leftmargin}{0.25in}\setlength{\topsep}{0.0in}}
\item The major version number of the library.
\end{list}
\vspace{2.8ex}

\noindent\textbf{TYPICAL USAGE}
\begin{list}{}{\setlength{\leftmargin}{0.25in}\setlength{\topsep}{0.0in}}
\item This function is typically used to guard against project build accidents
      where an unintended version of the library is linked in.  A typical
      approach is that the software will not function if an unexpected version
      of the library is present.
\end{list}
\vspace{2.8ex}

\noindent\textbf{INTERRUPT COMPATIBILITY}
\begin{list}{}{\setlength{\leftmargin}{0.25in}\setlength{\topsep}{0.0in}}
\item This function may be used from both non-ISR and ISR software.
\item This function is thread-safe.
\end{list}
\vspace{2.8ex}

\noindent\textbf{EXECUTION TIME}
\begin{list}{}{\setlength{\leftmargin}{0.25in}\setlength{\topsep}{0.0in}}
\item TBD.
\end{list}
\vspace{2.8ex}

\noindent\textbf{FUNCTION NAME MNEMONIC}
\begin{list}{}{\setlength{\leftmargin}{0.25in}\setlength{\topsep}{0.0in}}
\item \emph{Lib}: library.  \emph{Ver}: version.  \emph{Maj}: major.
\end{list}


%%%%%%%%%%%%%%%%%%%%%%%%%%%%%%%%%%%%%%%%%%%%%%%%%%%%%%%%%%%%%%%%%%%%%%%%%%%%%%%
%%%%%%%%%%%%%%%%%%%%%%%%%%%%%%%%%%%%%%%%%%%%%%%%%%%%%%%%%%%%%%%%%%%%%%%%%%%%%%%
%%%%%%%%%%%%%%%%%%%%%%%%%%%%%%%%%%%%%%%%%%%%%%%%%%%%%%%%%%%%%%%%%%%%%%%%%%%%%%%
\subsection[\emph{UcuMfLibVerMinMicRxx(\protect\mbox{\protect$\cdot$})}]
           {\emph{UcuMfLibVerMinMicRxx(\protect\mbox{\protect\boldmath $\cdot$})}}
%Section tag:  lvm0
\label{cmsc0:slvm0}

\index{UcuMfLibVerMinMicRxx()@\emph{UcuMfLibVerMinMicRxx($\cdot$)}}%

\noindent\textbf{PROTOTYPE}
\begin {list}{}{\setlength{\leftmargin}{0.25in}\setlength{\topsep}{0.0in}}
\item
\begin{verbatim}
UCU_UINT16 UcuMfLibVerMinMicRxx(void)
\end{verbatim}
\end{list}
\vspace{2.8ex}

\noindent\textbf{SYNOPSIS}
\begin{list}{}{\setlength{\leftmargin}{0.25in}\setlength{\topsep}{0.0in}}
\item
Returns the minor version number and micro version number of the library;
with the minor version number packed in the most significant byte of the
result and the micro version number packed in the least significant byte
of the result.
\end{list}
\vspace{2.8ex}

\noindent\textbf{INPUT}
\begin{list}{}{\setlength{\leftmargin}{0.5in}\setlength{\itemindent}{-0.25in}\setlength{\topsep}{0.0in}\setlength{\partopsep}{0.0in}}
\item None.
\end{list}
\vspace{2.8ex}

\noindent\textbf{OUTPUT}
\begin{list}{}{\setlength{\leftmargin}{0.25in}\setlength{\topsep}{0.0in}}
\item The minor version number and micro version number, packed as
      described in the synopsis.
\item The minor and micro version numbers are described more fully
      in \S\ref{ciov0:slcv0:slvn0}, p. \pageref{ciov0:slcv0:slvn0}.
\end{list}
\vspace{2.8ex}

\noindent\textbf{DETAILED DESCRIPTION}
\begin{list}{}{\setlength{\leftmargin}{0.25in}\setlength{\topsep}{0.0in}}
\item Returns the minor version number and micro version number of the library;
      with the minor version number packed in the most significant byte of the
      result and the micro version number packed in the least significant byte
      of the result.
\item The minor and micro version numbers are described more fully
      in \S\ref{ciov0:slcv0:slvn0}, p. \pageref{ciov0:slcv0:slvn0}.
\item As an example,
      if the library is version 2.5c, the integer returned would be $5 \times 2^{8} + 2$.
\end{list}
\vspace{2.8ex}

\noindent\textbf{TYPICAL USAGE}
\begin{list}{}{\setlength{\leftmargin}{0.25in}\setlength{\topsep}{0.0in}}
\item This function is typically used to guard against project build accidents
      where an unintended version of the library is linked in.  A typical
      approach is that the software will not function if an unexpected version
      of the library is present.
\end{list}
\vspace{2.8ex}

\noindent\textbf{INTERRUPT COMPATIBILITY}
\begin{list}{}{\setlength{\leftmargin}{0.25in}\setlength{\topsep}{0.0in}}
\item This function may be used from both non-ISR and ISR software.
\item This function is thread-safe.
\end{list}
\vspace{2.8ex}

\noindent\textbf{EXECUTION TIME}
\begin{list}{}{\setlength{\leftmargin}{0.25in}\setlength{\topsep}{0.0in}}
\item TBD.
\end{list}
\vspace{2.8ex}

\noindent\textbf{FUNCTION NAME MNEMONIC}
\begin{list}{}{\setlength{\leftmargin}{0.25in}\setlength{\topsep}{0.0in}}
\item \emph{Lib}: library.  \emph{Ver}: version.  \emph{Min}: minor.
      \emph{Mic}: micro.
\end{list}


%%%%%%%%%%%%%%%%%%%%%%%%%%%%%%%%%%%%%%%%%%%%%%%%%%%%%%%%%%%%%%%%%%%%%%%%%%%%%%%
%%%%%%%%%%%%%%%%%%%%%%%%%%%%%%%%%%%%%%%%%%%%%%%%%%%%%%%%%%%%%%%%%%%%%%%%%%%%%%%
%%%%%%%%%%%%%%%%%%%%%%%%%%%%%%%%%%%%%%%%%%%%%%%%%%%%%%%%%%%%%%%%%%%%%%%%%%%%%%%
\subsection[\emph{UcuMfLibVerCpuRxx(\protect\mbox{\protect$\cdot$})}]
           {\emph{UcuMfLibVerCpuRxx(\protect\mbox{\protect\boldmath $\cdot$})}}
%Section tag:  lvc0
\label{cmsc0:slvc0}

\index{UcuMfLibVerCpuRxx()@\emph{UcuMfLibVerCpuRxx($\cdot$)}}

\noindent\textbf{PROTOTYPE}
\begin {list}{}{\setlength{\leftmargin}{0.25in}\setlength{\topsep}{0.0in}}
\item
\begin{verbatim}
UCU_UINT16 UcuMfLibVerCpuRxx(void)
\end{verbatim}
\end{list}
\vspace{2.8ex}

\noindent\textbf{SYNOPSIS}
\begin{list}{}{\setlength{\leftmargin}{0.25in}\setlength{\topsep}{0.0in}}
\item Returns the CPU core / development tool chain identifier and the CPU core variant identifier,
      with the CPU core / development tool chain identifier in the most
      significant byte and the CPU core variant identifier in the least significant byte.
\end{list}
\vspace{2.8ex}

\noindent\textbf{INPUT}
\begin{list}{}{\setlength{\leftmargin}{0.5in}\setlength{\itemindent}{-0.25in}\setlength{\topsep}{0.0in}\setlength{\partopsep}{0.0in}}
\item None.
\end{list}
\vspace{2.8ex}

\noindent\textbf{OUTPUT}
\begin{list}{}{\setlength{\leftmargin}{0.25in}\setlength{\topsep}{0.0in}}
\item The CPU core / development tool chain identifier and the CPU core variant identifier, packed
      as described in the synopsis.  The CPU core / development tool chain identifiers
      that can be returned are listed in
      Table \ref{tbl:ciov0:sscv0:01}, p. \pageref{tbl:ciov0:sscv0:01}.

      Within the STM8/Cosmic combination, the CPU core variant codes are
      listed in 
      Table \ref{tbl:ciov0:sscv0:02}, p. \pageref{tbl:ciov0:sscv0:02}.

      Within the CPU08/Cosmic combination, the CPU core variant codes are
      listed in 
      Table \ref{tbl:ciov0:sscv0:03}, p. \pageref{tbl:ciov0:sscv0:03}.
\end{list}
\vspace{2.8ex}

\noindent\textbf{TYPICAL USAGE}
\begin{list}{}{\setlength{\leftmargin}{0.25in}\setlength{\topsep}{0.0in}}
\item This function is typically used to guard against project build accidents
      where an unintended version of the library is linked in.  A typical
      approach is that the software will not function if an unexpected version
      of the library is present.
\end{list}
\vspace{2.8ex}

\noindent\textbf{INTERRUPT COMPATIBILITY}
\begin{list}{}{\setlength{\leftmargin}{0.25in}\setlength{\topsep}{0.0in}}
\item This function may be used from both non-ISR and ISR software.
\item This function is thread-safe.
\end{list}
\vspace{2.8ex}

\noindent\textbf{EXECUTION TIME}
\begin{list}{}{\setlength{\leftmargin}{0.25in}\setlength{\topsep}{0.0in}}
\item TBD.
\end{list}
\vspace{2.8ex}

\noindent\textbf{FUNCTION NAME MNEMONIC}
\begin{list}{}{\setlength{\leftmargin}{0.25in}\setlength{\topsep}{0.0in}}
\item \emph{Lib}: library.  \emph{Ver}: version.  \emph{Cpu}: CPU.
\end{list}


%%%%%%%%%%%%%%%%%%%%%%%%%%%%%%%%%%%%%%%%%%%%%%%%%%%%%%%%%%%%%%%%%%%%%%%%%%%%%%%
%%%%%%%%%%%%%%%%%%%%%%%%%%%%%%%%%%%%%%%%%%%%%%%%%%%%%%%%%%%%%%%%%%%%%%%%%%%%%%%
%%%%%%%%%%%%%%%%%%%%%%%%%%%%%%%%%%%%%%%%%%%%%%%%%%%%%%%%%%%%%%%%%%%%%%%%%%%%%%%
\subsection[\emph{UcuMfLibVerCmpRxx(\protect\mbox{\protect$\cdot$})}]
           {\emph{UcuMfLibVerCmpRxx(\protect\mbox{\protect\boldmath $\cdot$})}}
%Section tag:  lcp0
\label{cmsc0:slcp0}

\index{UcuMfLibVerCmpRxx()@\emph{UcuMfLibVerCmpRxx($\cdot$)}}%

\noindent\textbf{PROTOTYPE}
\begin {list}{}{\setlength{\leftmargin}{0.25in}\setlength{\topsep}{0.0in}}
\item
\begin{verbatim}
UCU_BOOLEAN UcuMfLibVerCpuRxx(
                             UCU_UINT16 in_majorversion,
                             UCU_UINT8  in_minorversion,
                             UCU_UINT8  in_microversion,
                             UCU_UINT8  in_cpucore,
                             UCU_UINT8  in_cpucorevariant
                             )
\end{verbatim}
\end{list}
\vspace{2.8ex}

\noindent\textbf{SYNOPSIS}
\begin{list}{}{\setlength{\leftmargin}{0.25in}\setlength{\topsep}{0.0in}}
\item
Compares the passed version information against the actual version
of the library and returns UCU\_TRUE if the passed information matches
the actual library version or UCU\_FALSE otherwise.
\end{list}
\vspace{2.8ex}

\noindent\textbf{INPUTS}
\begin{list}{}{\setlength{\leftmargin}{0.5in}\setlength{\itemindent}{-0.25in}\setlength{\topsep}{0.0in}\setlength{\partopsep}{0.0in}}
\item \emph{\textbf{in\_majorversion}}\\
      The major version number used to compare against the actual version
      number.  A value of $2^{16}-1$ will cause the major version number
      to be ignored in the comparison.
\item \emph{\textbf{in\_minorversion}}\\
      The minor version number used to compare against the actual version
      number.  A value of $2^{8}-1$ will cause the minor version number
      to be ignored in the comparison.
\item \emph{\textbf{in\_microversion}}\\
      The micro version number used to compare against the actual version
      number.  A value of $2^{8}-1$ will cause the micro version number
      to be ignored in the comparison.
\item \emph{\textbf{in\_cpucore}}\\
      The CPU core identifier used to compare against the actual CPU core
      identifier for which the library was compiled.
      A value of $2^{8}-1$ will cause the CPU core identifier
      to be ignored in the comparison.
\item \emph{\textbf{in\_cpucorevariant}}\\
      The CPU core variant identifier used to compare against the actual CPU core
      variant identifier for which the library was compiled.
      A value of $2^{8}-1$ will cause the CPU core variant identifier
      to be ignored in the comparison.
\end{list}
\vspace{2.8ex}

\noindent\textbf{OUTPUTS}
\begin{list}{}{\setlength{\leftmargin}{0.25in}\setlength{\topsep}{0.0in}}
\item UCU\_TRUE if the version information of the library matches
      the information provided, or UCU\_FALSE otherwise.
\end{list}
\vspace{2.8ex}

\noindent\textbf{EXCEPTION CASES}
\begin{list}{}{\setlength{\leftmargin}{0.25in}\setlength{\topsep}{0.0in}}
\item If all of the inputs are set to the ``ignore'' values,
      the function will return UCU\_TRUE.
\end{list}
\vspace{2.8ex}

\noindent\textbf{TYPICAL USAGE}
\begin{list}{}{\setlength{\leftmargin}{0.25in}\setlength{\topsep}{0.0in}}
\item This function is most often used near reset to ensure that the
      version of \emph{\productname{}} linked into the software is
      the expected version.  A typical approach would be that the
      embedded software does not function if the library linked in is
      not the expected version for the expected CPU.
\item Figure \ref{fig:cmsc0:slcp0:00} (p. \pageref{fig:cmsc0:slcp0:00})
      shows a typical usage of the function.  The code snippet 
      shown in the figure checks both that the \texttt{UCULIB.H} file
      is consistent with the library, and that library is the expected
      version built for the expected CPU.
\end{list}
\vspace{2.8ex}

\begin{figure}
\begin{small}
\begin{verbatim}
//Halt the compilation if the UCULIB .H file isn't the expected
//version.
#if (UCU_LIBVER_MAJOR != 0) || (UCU_LIBVER_MINOR != 1) || \
    (UCU_LIBVER_MICRO != 0) ||                            \
    (UCU_LIBVER_CPUCORE != UCU_CPUCORE_STM8) ||           \
    (UCU_LIBVER_CPUCOREVAR != UCU_CPUCOREVAR_STM8_BASE)
   //The UCULIB .H file isn't the right version.  Error out.
   #error "UCULIB .H file is unexpected version."
#endif

//If the .H file doesn't match the actual library, this
//should prevent operation of the product.
if (! UcuMfLibVerCmpRxx(UCU_LIBVER_MAJOR, 
                        UCU_LIBVER_MINOR, 
                        UCU_LIBVER_MICRO, 
                        UCU_LIBVER_CPUCORE, 
                        UCU_LIBVER_CPUCOREVAR))
   {
   while(1)
      ;
   }
\end{verbatim}
\end{small}
\caption{Typical Usage of the \emph{UcuMfLibVerCmpRxx($\cdot$)} Function}
\label{fig:cmsc0:slcp0:00}
\end{figure}

\noindent\textbf{INTERRUPT COMPATIBILITY}
\begin{list}{}{\setlength{\leftmargin}{0.25in}\setlength{\topsep}{0.0in}}
\item This function may be used from both non-ISR and ISR software.
\item This function is thread-safe.
\end{list}
\vspace{2.8ex}

\noindent\textbf{EXECUTION TIME}
\begin{list}{}{\setlength{\leftmargin}{0.25in}\setlength{\topsep}{0.0in}}
\item TBD.
\end{list}
\vspace{2.8ex}

\noindent\textbf{FUNCTION NAME MNEMONIC}
\begin{list}{}{\setlength{\leftmargin}{0.25in}\setlength{\topsep}{0.0in}}
\item \emph{Lib}: library.  \emph{Ver}: version.  \emph{Cmp}: compare.
\end{list}


%%%%%%%%%%%%%%%%%%%%%%%%%%%%%%%%%%%%%%%%%%%%%%%%%%%%%%%%%%%%%%%%%%%%%%%%%%%%%%%
%%%%%%%%%%%%%%%%%%%%%%%%%%%%%%%%%%%%%%%%%%%%%%%%%%%%%%%%%%%%%%%%%%%%%%%%%%%%%%%
%%%%%%%%%%%%%%%%%%%%%%%%%%%%%%%%%%%%%%%%%%%%%%%%%%%%%%%%%%%%%%%%%%%%%%%%%%%%%%%
\section{CPU Register Manipulation Functions}
\label{cmsc0:scpu0}


%%%%%%%%%%%%%%%%%%%%%%%%%%%%%%%%%%%%%%%%%%%%%%%%%%%%%%%%%%%%%%%%%%%%%%%%%%%%%%%
%%%%%%%%%%%%%%%%%%%%%%%%%%%%%%%%%%%%%%%%%%%%%%%%%%%%%%%%%%%%%%%%%%%%%%%%%%%%%%%
%%%%%%%%%%%%%%%%%%%%%%%%%%%%%%%%%%%%%%%%%%%%%%%%%%%%%%%%%%%%%%%%%%%%%%%%%%%%%%%
\subsection[\emph{UcuMfCpuCcrGetRxx(\protect\mbox{\protect$\cdot$})}]
           {\emph{UcuMfCpuCcrGetRxx(\protect\mbox{\protect\boldmath $\cdot$})}}
\label{cmsc0:scpu0:sccg0}

\index{UcuMfCpuCcrGetRxx()@\emph{UcuMfCpuCcrGetRxx($\cdot$)}}%

\noindent\textbf{PROTOTYPE}
\begin {list}{}{\setlength{\leftmargin}{0.25in}\setlength{\topsep}{0.0in}}
\item
\begin{verbatim}
UCU_CPU_CCR UcuMfCpuCcrGetRxx(void)
\end{verbatim}
\end{list}
\vspace{2.8ex}

\noindent\textbf{SYNOPSIS}
\begin{list}{}{\setlength{\leftmargin}{0.25in}\setlength{\topsep}{0.0in}}
\item
Returns the value of the CPU condition code register.
\end{list}
\vspace{2.8ex}

\noindent\textbf{INPUT}
\begin{list}{}{\setlength{\leftmargin}{0.5in}\setlength{\itemindent}{-0.25in}\setlength{\topsep}{0.0in}\setlength{\partopsep}{0.0in}}
\item None.
\end{list}
\vspace{2.8ex}

\noindent\textbf{OUTPUT}
\begin{list}{}{\setlength{\leftmargin}{0.25in}\setlength{\topsep}{0.0in}}
\item The value of the CPU condition code register.
\item Because this is a function and a function call is involved,
      certain bits or bitfields in the condition code register may not
      be meaningful in a higher-level language program.  However, some
      bits or bitfields may be meaningful, especially those involving
      the interrupt mask.
\end{list}
\vspace{2.8ex}

\noindent\textbf{TYPICAL USAGE}
\begin{list}{}{\setlength{\leftmargin}{0.25in}\setlength{\topsep}{0.0in}}
\item This function is typically used to test the interrupt mask bits.
\end{list}
\vspace{2.8ex}

\noindent\textbf{INTERRUPT COMPATIBILITY}
\begin{list}{}{\setlength{\leftmargin}{0.25in}\setlength{\topsep}{0.0in}}
\item This function may be used from both non-ISR and ISR software.
\item This function is thread-safe.
\end{list}
\vspace{2.8ex}

\noindent\textbf{EXECUTION TIME}
\begin{list}{}{\setlength{\leftmargin}{0.25in}\setlength{\topsep}{0.0in}}
\item TBD.
\end{list}
\vspace{2.8ex}

\noindent\textbf{FUNCTION NAME MNEMONIC}
\begin{list}{}{\setlength{\leftmargin}{0.25in}\setlength{\topsep}{0.0in}}
\item \emph{Cpu}: CPU.  \emph{Ccr}: condition code register.  \emph{Get}: get.
\end{list}


%%%%%%%%%%%%%%%%%%%%%%%%%%%%%%%%%%%%%%%%%%%%%%%%%%%%%%%%%%%%%%%%%%%%%%%%%%%%%%%
%%%%%%%%%%%%%%%%%%%%%%%%%%%%%%%%%%%%%%%%%%%%%%%%%%%%%%%%%%%%%%%%%%%%%%%%%%%%%%%
%%%%%%%%%%%%%%%%%%%%%%%%%%%%%%%%%%%%%%%%%%%%%%%%%%%%%%%%%%%%%%%%%%%%%%%%%%%%%%%
\subsection[\emph{UcuMfCpuCcrSetRxx(\protect\mbox{\protect$\cdot$})}]
           {\emph{UcuMfCpuCcrSetRxx(\protect\mbox{\protect\boldmath $\cdot$})}}
\label{cmsc0:scpu0:sccs0}

\index{UcuMfCpuCcrSetRxx()@\emph{UcuMfCpuCcrSetRxx($\cdot$)}}%

\noindent\textbf{PROTOTYPE}
\begin {list}{}{\setlength{\leftmargin}{0.25in}\setlength{\topsep}{0.0in}}
\item
\begin{verbatim}
void UcuMfCpuCcrSetRxx(UCU_CPU_CCR in_ccr)
\end{verbatim}
\end{list}
\vspace{2.8ex}

\noindent\textbf{SYNOPSIS}
\begin{list}{}{\setlength{\leftmargin}{0.25in}\setlength{\topsep}{0.0in}}
\item Atomically sets the CPU condition code register to the value specified.
\item \emph{Atomic} in this sense means that the new condition code register
      value is set in an atomic instruction or atomic sequence of instructions.  This
      is particularly important for the interrupt mask bits (to ensure that an
      interrupt may not occur with these bits set to values that are not the original
      values and not the \emph{in\_ccr} value.
\end{list}
\vspace{2.8ex}

\noindent\textbf{INPUT}
\begin{list}{}{\setlength{\leftmargin}{0.5in}\setlength{\itemindent}{-0.25in}\setlength{\topsep}{0.0in}\setlength{\partopsep}{0.0in}}
\item \emph{\textbf{in\_ccr}}\\
      The value to which the CPU condition code register should be atomically set.
\end{list}
\vspace{2.8ex}

\noindent\textbf{OUTPUT}
\begin{list}{}{\setlength{\leftmargin}{0.25in}\setlength{\topsep}{0.0in}}
\item None.
\end{list}
\vspace{2.8ex}

\noindent\textbf{TYPICAL USAGE}
\begin{list}{}{\setlength{\leftmargin}{0.25in}\setlength{\topsep}{0.0in}}
\item This function is typically used to atomically set the interrupt mask bits.
\item Because the compiler generally assumes that CCR bits are not preserved
      across a function call, there is no advantage to providing this function
      in a form that modifies only the interrupt mask bits (the modifications
      to the other bits will be discarded).
\end{list}
\vspace{2.8ex}

\noindent\textbf{INTERRUPT COMPATIBILITY}
\begin{list}{}{\setlength{\leftmargin}{0.25in}\setlength{\topsep}{0.0in}}
\item This function may be used from both non-ISR and ISR software.
\item This function is thread-safe.
\end{list}
\vspace{2.8ex}

\noindent\textbf{EXECUTION TIME}
\begin{list}{}{\setlength{\leftmargin}{0.25in}\setlength{\topsep}{0.0in}}
\item TBD.
\end{list}
\vspace{2.8ex}

\noindent\textbf{FUNCTION NAME MNEMONIC}
\begin{list}{}{\setlength{\leftmargin}{0.25in}\setlength{\topsep}{0.0in}}
\item \emph{Cpu}: CPU.  \emph{Ccr}: condition code register.  \emph{Set}: set.
\end{list}


%%%%%%%%%%%%%%%%%%%%%%%%%%%%%%%%%%%%%%%%%%%%%%%%%%%%%%%%%%%%%%%%%%%%%%%%%%%%%%%
%%%%%%%%%%%%%%%%%%%%%%%%%%%%%%%%%%%%%%%%%%%%%%%%%%%%%%%%%%%%%%%%%%%%%%%%%%%%%%%
%%%%%%%%%%%%%%%%%%%%%%%%%%%%%%%%%%%%%%%%%%%%%%%%%%%%%%%%%%%%%%%%%%%%%%%%%%%%%%%
\subsection[\emph{UcuMfCpuSptrPcGetRxx(\protect\mbox{\protect$\cdot$})}]
           {\emph{UcuMfCpuSptrPcGetRxx(\protect\mbox{\protect\boldmath $\cdot$})}}
\label{cmsc0:scpu0:sspg0}

\index{UcuMfCpuSptrPcGetRxx()@\emph{UcuMfCpuSptrPcGetRxx($\cdot$)}}%

\noindent\textbf{PROTOTYPE}
\begin {list}{}{\setlength{\leftmargin}{0.25in}\setlength{\topsep}{0.0in}}
\item
\begin{verbatim}
UCU_CPU_SPTR_PC UcuMfCpuSptrPcGetRxx(void)
\end{verbatim}
\end{list}
\vspace{2.8ex}

\noindent\textbf{SYNOPSIS}
\begin{list}{}{\setlength{\leftmargin}{0.25in}\setlength{\topsep}{0.0in}}
\item
Returns the value of the program counter stack pointer at the time 
the function is called.
\end{list}
\vspace{2.8ex}

\noindent\textbf{INPUT}
\begin{list}{}{\setlength{\leftmargin}{0.5in}\setlength{\itemindent}{-0.25in}\setlength{\topsep}{0.0in}\setlength{\partopsep}{0.0in}}
\item None.
\end{list}
\vspace{2.8ex}

\noindent\textbf{OUTPUT}
\begin{list}{}{\setlength{\leftmargin}{0.25in}\setlength{\topsep}{0.0in}}
\item The value of the program counter stack pointer at the time the
      function is called.
\item In order to calculate the value of the program counter stack pointer
      at the time the function is called,
      the stack pointer is adjusted by the number of bytes required
      to call the function.  This offset can vary with the processor and
      memory model.
\end{list}
\vspace{2.8ex}

\noindent\textbf{TYPICAL USAGE}
\begin{list}{}{\setlength{\leftmargin}{0.25in}\setlength{\topsep}{0.0in}}
\item This function is typically used to assist in debugging 
      programs, or to test whether a called function corrupts the stack pointer.
\end{list}
\vspace{2.8ex}

\noindent\textbf{INTERRUPT COMPATIBILITY}
\begin{list}{}{\setlength{\leftmargin}{0.25in}\setlength{\topsep}{0.0in}}
\item This function may be used from both non-ISR and ISR software.
\item This function is thread-safe.
\end{list}
\vspace{2.8ex}

\noindent\textbf{EXECUTION TIME}
\begin{list}{}{\setlength{\leftmargin}{0.25in}\setlength{\topsep}{0.0in}}
\item TBD.
\end{list}
\vspace{2.8ex}

\noindent\textbf{FUNCTION NAME MNEMONIC}
\begin{list}{}{\setlength{\leftmargin}{0.25in}\setlength{\topsep}{0.0in}}
\item \emph{Cpu}:  CPU.  
      \emph{Sptr}: stack pointer.  
      \emph{Pc}:   program counter.  
      \emph{Get}:  get.
\end{list}


%%%%%%%%%%%%%%%%%%%%%%%%%%%%%%%%%%%%%%%%%%%%%%%%%%%%%%%%%%%%%%%%%%%%%%%%%%
\noindent\begin{figure}[!b]
\noindent\rule[-0.25in]{\textwidth}{1pt}
\begin{tiny}
\begin{verbatim}
$RCSfile: c_msc0.tex,v $
$Source: /home/dashley/cvsrep/uculib01/uculib01/doc/manual/c_msc0/c_msc0.tex,v $
$Revision: 1.11 $
$Author: dashley $
$Date: 2010/06/11 15:38:58 $
\end{verbatim}
\end{tiny}
\noindent\rule[0.25in]{\textwidth}{1pt}
\end{figure}

%%%%%%%%%%%%%%%%%%%%%%%%%%%%%%%%%%%%%%%%%%%%%%%%%%%%%%%%%%%%%%%%%%%%%%%%%%%%%%%
%$Log: c_msc0.tex,v $
%Revision 1.11  2010/06/11 15:38:58  dashley
%Addition of UcuMfCpuCcrSetRxx() function.
%
%Revision 1.10  2010/02/14 02:37:44  dashley
%Function names changed.
%
%Revision 1.9  2010/02/14 02:20:54  dashley
%Two functions added.
%
%Revision 1.8  2010/02/04 17:51:12  dashley
%Exception case explanation added.
%
%Revision 1.7  2010/02/04 16:16:23  dashley
%Documentation updates.
%
%Revision 1.6  2010/01/28 21:18:32  dashley
%a)Chapter start quotes removed.
%b)Aesthetic comment line added at the bottom of most files.
%
%Revision 1.5  2010/01/27 21:52:15  dashley
%Edits.
%%%%%%%%%%%%%%%%%%%%%%%%%%%%%%%%%%%%%%%%%%%%%%%%%%%%%%%%%%%%%%%%%%%%%%%%%%%%%%%
%End of $RCSfile: c_msc0.tex,v $.
%%%%%%%%%%%%%%%%%%%%%%%%%%%%%%%%%%%%%%%%%%%%%%%%%%%%%%%%%%%%%%%%%%%%%%%%%%%%%%%

