%$Header: /home/dashley/cvsrep/uculib01/uculib01/doc/manual/c_tbg0/c_tbg0.tex,v 1.12 2010/04/05 15:13:50 dashley Exp $

\chapter{Technical Background}        
\label{ctbg0}

%%%%%%%%%%%%%%%%%%%%%%%%%%%%%%%%%%%%%%%%%%%%%%%%%%%%%%%%%%%%%%%%%%%%%%%%%%%%%%%
%%%%%%%%%%%%%%%%%%%%%%%%%%%%%%%%%%%%%%%%%%%%%%%%%%%%%%%%%%%%%%%%%%%%%%%%%%%%%%%
%%%%%%%%%%%%%%%%%%%%%%%%%%%%%%%%%%%%%%%%%%%%%%%%%%%%%%%%%%%%%%%%%%%%%%%%%%%%%%%
\section{Introduction}
%Section tag:  int0
\label{ctbg0:sint0}


%%%%%%%%%%%%%%%%%%%%%%%%%%%%%%%%%%%%%%%%%%%%%%%%%%%%%%%%%%%%%%%%%%%%%%%%%%%%%%%
%%%%%%%%%%%%%%%%%%%%%%%%%%%%%%%%%%%%%%%%%%%%%%%%%%%%%%%%%%%%%%%%%%%%%%%%%%%%%%%
%%%%%%%%%%%%%%%%%%%%%%%%%%%%%%%%%%%%%%%%%%%%%%%%%%%%%%%%%%%%%%%%%%%%%%%%%%%%%%%
\section{Interrupt Compatibility Issues}
%Section tag:  ici0
\label{ctbg0:sici0}


%%%%%%%%%%%%%%%%%%%%%%%%%%%%%%%%%%%%%%%%%%%%%%%%%%%%%%%%%%%%%%%%%%%%%%%%%%%%%%%
%%%%%%%%%%%%%%%%%%%%%%%%%%%%%%%%%%%%%%%%%%%%%%%%%%%%%%%%%%%%%%%%%%%%%%%%%%%%%%%
%%%%%%%%%%%%%%%%%%%%%%%%%%%%%%%%%%%%%%%%%%%%%%%%%%%%%%%%%%%%%%%%%%%%%%%%%%%%%%%
\section{Linear Software Timers}
%Section tag:  lst0
\label{ctbg0:slst0}

Several of the block memory functions described in
Chapter \ref{cbmf0} are designed primarily to 
implement what we call here a 
\index{software timer}\index{linear software timer}\emph{software timer} (although these functions
may have other uses).

We define a \index{software timer}\index{linear software timer}software timer as an unsigned integer
of any length convenient to the machine that:

\begin{itemize}
\item Is decremented or incremented at a constant rate,\footnote{It is also
      possible to decrement (or increment) software timers at a non-constant rate to
      approximate equations of the form $x(k+1) = \overline{\alpha} x(k)$, 
      $\overline{\alpha} < 1$ (or related
      equations), but we don't discuss schemes of this type here.  The advantage
      of such a scheme would be to extend the maximum time period of the timer
      while still keeping good precision at low values of time.  We would call
      such software timers \index{non-linear software timer}\index{software timer}%
      \emph{non-linear} software timers to differentiate them
      from the timers described here.}
      but not
      below or above the minimum or maximum value of an unsigned
      integer.
\item Is set and tested by one process (the application) but 
      is decremented or incremented by another process (typically
      system software).
\end{itemize}

Typically, linear software timers are grouped together as arrays of 
unsigned integers that are decremented or incremented at the same rate (hence the
efficiency of the block manipulation described in Chapter \ref{cbmf0}).
Most commonly, the rates at which groups of timers may be decremented or
incremented are provided in binary decades (but 1-2-5 decades are also
not uncommon).

Linear software timers are the most efficient known way to manipulate time
in stateful logic where the logic is implemented as code (rather than data).
The efficiency comes about because:

\begin{itemize}
\item The timers are decremented and incremented centrally using
      block memory operations (saves FLASH bulk
      and CPU cycles).
\item The arrangement of timers in binary decades or 1-2-5 decades
      allows the usage of smaller data types (at the expense of precision)
      than would be possible if time were manipulated only in terms of one
      base quantum.
\item The test for an expired timer ($t=0$) often compiles well because
      tests against zero are less expensive than tests against other constants
      in CPU instruction sets.
\end{itemize}

In most of the discussion which follows, we confine the discussion to
the case of software timers that are decremented, but not below zero.
(Software timers that are incremented are less common and have different
applications.)

The arrangement of software timers in most applications is in terms of
a base quantum, which we denote $\sigma$\@.  $\sigma$=1 millisecond is
a common choice.

In a binary decade arrangement, the quantum used to decrement the
different arrays is $\sigma 2^q$, where $q$ is the ``tier'' of the array of
software timers.  If $\sigma$=1 millisecond, ``Tier 0'' software timers
are decremented every millisecond, ``Tier 1'' software timers are decremented
every two milliseconds, ``Tier 2'' software timers are decremented
every four milliseconds, and so on.

Note that if a software timer (at tier $q$) is set to a value $n$
and then tested periodically to determine if it has reached zero, the
actual amount of time $t$ that may have elapsed is given by\footnote{In
order to obtain (\ref{eq:ctbg0:slst0:01}), we make the assumption that
the process that tests the software timer runs at least as often as the
software timer is decremented.  There are scheduling scenarios that add
yet more to the error, and these are not developed here.}

\begin{equation}
\label{eq:ctbg0:slst0:01}
(n - 1) \sigma 2^q 
<
t
\leq
n \sigma 2^q .
\end{equation}

\noindent{}(\ref{eq:ctbg0:slst0:01}) comes about because it is possible
that the timer was set to $n$ just before it was decremented.

We propose to choose

\begin{equation}
\label{eq:ctbg0:slst0:02}
n = \left\lfloor {\frac{\alpha}{\sigma 2^q}} \right\rfloor .
\end{equation}

%%%%%%%%%%%%%%%%%%%%%%%%%%%%%%%%%%%%%%%%%%%%%%%%%%%%%%%%%%%%%%%%%%%%%%%%%%%%%%%
\noindent\rule{\textwidth}{2pt}

TODO:

\begin{enumerate}
\item Break into subsections.
\item Develop scheduling assumptions that lead to (\ref{eq:ctbg0:slst0:01}) and friends.
\item Show relationship with timed automata.
\item Show state-space and complexity reduction advantages.
\end{enumerate}


%%%%%%%%%%%%%%%%%%%%%%%%%%%%%%%%%%%%%%%%%%%%%%%%%%%%%%%%%%%%%%%%%%%%%%%%%%%%%%%
%%%%%%%%%%%%%%%%%%%%%%%%%%%%%%%%%%%%%%%%%%%%%%%%%%%%%%%%%%%%%%%%%%%%%%%%%%%%%%%
%%%%%%%%%%%%%%%%%%%%%%%%%%%%%%%%%%%%%%%%%%%%%%%%%%%%%%%%%%%%%%%%%%%%%%%%%%%%%%%
\section{Square Root Extraction (Integer)}
%Section tag:  sre0
\label{ctbg0:ssre0}

This section presents information about the extraction of
square roots over an integer domain and integer range.  Usually, the function
of interest is

\begin{equation}
\label{eq:ctbg0:ssre0:01}
f(x) = \left\lfloor \sqrt{x} \right\rfloor .
\end{equation}

Other functions of interest can generally be implemented in terms
of (\ref{eq:ctbg0:ssre0:01}).

If a square root with information after the radix point is desired, the argument $x$ can be premultiplied
by the square of the reciprocal of the precision after the radix point.  For example,
a good approximation to $10 \sqrt{x}$ is

\begin{equation}
\label{eq:ctbg0:ssre0:02}
10 \sqrt{x} \approx g(x) = \left\lfloor 10 \sqrt{x} \right\rfloor
                         = \left\lfloor \sqrt{100 x} \right\rfloor ,
\end{equation}

\noindent{}which can be implemented in terms of (\ref{eq:ctbg0:ssre0:01})\@.
(\ref{eq:ctbg0:ssre0:02}) provides a result that is human-friendly (i.e. 453 is
interpreted as 45.3), but the radix point of the result is not positioned between
any two bits.  A more typical approach might be to premultiply by an even
power of two so as to place the radix point of the result between two bits.
For example,

\begin{equation}
\label{eq:ctbg0:ssre0:02b}
2^8 \sqrt{x}       \approx \left\lfloor 2^8 \sqrt{x} \right\rfloor
                         = \left\lfloor \sqrt{2^{16} x} \right\rfloor 
\end{equation}

\noindent{}will calculate the square root of an integer
with eight bits after the radix point.

Similarly, if rounding to the nearest integer is desired, note that\footnote{To
prove (\ref{eq:ctbg0:ssre0:03}), consider four cases.  \emph{Case 1:} If the fractional part of
$\sqrt{x}$ is 0, then $\lfloor \sqrt{4x} \rfloor = \lfloor 2 \sqrt{x} \rfloor$
will be an integer, the result will be exact, and (\ref{eq:ctbg0:ssre0:03}) holds.
\emph{Case 2:} If the fractional part of
$\sqrt{x}$ is less than 0.5, then $\lfloor \sqrt{4x} \rfloor = \lfloor 2 \sqrt{x} \rfloor$
will be an even integer, the square root will be rounded down, and (\ref{eq:ctbg0:ssre0:03}) holds.
\emph{Case 3:} If the fractional part of
$\sqrt{x}$ is equal to 0.5, then $\lfloor \sqrt{4x} \rfloor = \lfloor 2 \sqrt{x} \rfloor$
will be an odd integer, the square root will be rounded up, and (\ref{eq:ctbg0:ssre0:03}) holds.
\emph{Case 4:} If the fractional part of
$\sqrt{x}$ is greater than 0.5, then $\lfloor \sqrt{4x} \rfloor = \lfloor 2 \sqrt{x} \rfloor$
will be an odd integer, the square root will be rounded up, and (\ref{eq:ctbg0:ssre0:03}) holds.}

\begin{equation}
\label{eq:ctbg0:ssre0:03}
h(x) = \left\lfloor \sqrt{x} + 0.5 \right\rfloor
     = \left\lfloor \frac{\left\lfloor \sqrt{4 x} \right\rfloor + 1}{2} \right\rfloor ,
\end{equation}

\noindent{}which can also be easily implemented in terms of (\ref{eq:ctbg0:ssre0:01}).


%%%%%%%%%%%%%%%%%%%%%%%%%%%%%%%%%%%%%%%%%%%%%%%%%%%%%%%%%%%%%%%%%%%%%%%%%%%%%%%
%%%%%%%%%%%%%%%%%%%%%%%%%%%%%%%%%%%%%%%%%%%%%%%%%%%%%%%%%%%%%%%%%%%%%%%%%%%%%%%
%%%%%%%%%%%%%%%%%%%%%%%%%%%%%%%%%%%%%%%%%%%%%%%%%%%%%%%%%%%%%%%%%%%%%%%%%%%%%%%
\subsection{Summation of Odd Integers}
\label{ctbg0:ssre0:ssoi0}

Consider the square of a non-negative integer $n$, $n^2$.  The distance from
$n^2$ to the square of the next larger integer is

\begin{equation}
\label{eq:ctbg0:ssre0:ssoi0:01}
(n+1)^2 - n^2 = 2n + 1 .
\end{equation}

$2n+1, n \in \mathbb{Z}^+$ is the set of odd non-negative integers.
It follows directly that the set of squared integers can be formed by
successively adding odd integers, i.e.

\begin{eqnarray}
\nonumber 1^2 =   1   & = & 1 \\
          2^2 =   4   & = & 1 + 3 \\
\nonumber 3^2 =   9   & = & 1 + 3 + 5 \\
\nonumber 4^2 =  16   & = & 1 + 3 + 5 + 7 \\
\nonumber             & \ldots{} &
\end{eqnarray}

One possible integer square root algorithm would involve summing consecutive
odd integers until the argument is exceeded.
This algorithm is not developed further because it would be impractical to
calculate the square root of any integers except small integers using this
method.


%%%%%%%%%%%%%%%%%%%%%%%%%%%%%%%%%%%%%%%%%%%%%%%%%%%%%%%%%%%%%%%%%%%%%%%%%%%%%%%
%%%%%%%%%%%%%%%%%%%%%%%%%%%%%%%%%%%%%%%%%%%%%%%%%%%%%%%%%%%%%%%%%%%%%%%%%%%%%%%
%%%%%%%%%%%%%%%%%%%%%%%%%%%%%%%%%%%%%%%%%%%%%%%%%%%%%%%%%%%%%%%%%%%%%%%%%%%%%%%
\subsection{Bit-By-Bit Bisection}
\label{ctbg0:ssre0:sbis0}

Square roots of an integer can be found using bit-by-bit bisection.
In this method, starting with the most significant bit of the result, each bit
is assumed `1' and a trial square is calculated.  If the trial square is no
larger than the argument, that bit of the result is `1'.  However, if the trial
square is larger than the argument, that bit of the result is `0'.

\begin{tiny}
\begin{verbatim}
Essentially, it is bit-by-bit trial squaring.  "Bisection" because 
each bit of the square root cuts the interval (in which the square root
might lie) in half.
 
Two of the posters in comp.arch.embedded indicated that "bisection" can be
economized further by using the identity:
 
(n + 2^k)^2 = n^2 + n*2^(k+1) + 2^(2*k)
 
The "n^2" term is significant because it means that if you have the
previous trial square and you are considering setting bit "k", you can
compute the next trial square from the previous square using only shifting
and adding.
 
If you carry this to the extreme, because you initially assume no bits
are set, you start with n = n**2 = 0, and the first trial square does
not involve multiplication.  So you can implement the entire square
algorithm without multiplying, even once.
 
Anyway, applying all possible optimizations, one gets this:
 
http://www.uculib.com/vvcuculib01/viewvc.cgi/uculib01/src/stm8/
cosmic/modx0/atu24sqrtfrxn/devel/alg_eval01.c?revision=1.7&view=markup
 
or this in the 32-bit case:
 
unsigned int sqrt32(unsigned long arg)
   {
   unsigned long two_2k_mask;
   unsigned long trial_square;
   unsigned long trial_square_prev;
   unsigned long square_root_ls_ip1;

   two_2k_mask = 0x40000000UL;
    trial_square      = 0;
   trial_square_prev = 0;
   square_root_ls_ip1 = 0;

   while(1)
      {
      trial_square =    (trial_square_prev) 
              + (square_root_ls_ip1)
              + (two_2k_mask);

      square_root_ls_ip1 >>= 1; 

      if (arg >= trial_square)
        {
        square_root_ls_ip1 |= two_2k_mask;
        trial_square_prev  =  trial_square; 
         }

      if (two_2k_mask == 1)
         break;

      two_2k_mask  >>= 2;
      }
   
   return(square_root_ls_ip1);
   }

If you look at the loop, there really isn't much there.  There will be
16 iterations.  The test against "1" to exit the loop can be implemented
as a single byte test (because a single "1" is being right-shifted, "1"
as the value of the least significant byte means that the other three
bytes are zero).
 
Temporary storage requirements are about 4 times the size of the input
argument.  So, for a 32-bit argument, that would be 16 bytes.
\end{verbatim}
\end{tiny}


%%%%%%%%%%%%%%%%%%%%%%%%%%%%%%%%%%%%%%%%%%%%%%%%%%%%%%%%%%%%%%%%%%%%%%%%%%%%%%%
%%%%%%%%%%%%%%%%%%%%%%%%%%%%%%%%%%%%%%%%%%%%%%%%%%%%%%%%%%%%%%%%%%%%%%%%%%%%%%%
%%%%%%%%%%%%%%%%%%%%%%%%%%%%%%%%%%%%%%%%%%%%%%%%%%%%%%%%%%%%%%%%%%%%%%%%%%%%%%%
\subsection{The Babylonian Method}
\label{ctbg0:ssre0:sbab0}

TBD.


%%%%%%%%%%%%%%%%%%%%%%%%%%%%%%%%%%%%%%%%%%%%%%%%%%%%%%%%%%%%%%%%%%%%%%%%%%%%%%%
%%%%%%%%%%%%%%%%%%%%%%%%%%%%%%%%%%%%%%%%%%%%%%%%%%%%%%%%%%%%%%%%%%%%%%%%%%%%%%%
%%%%%%%%%%%%%%%%%%%%%%%%%%%%%%%%%%%%%%%%%%%%%%%%%%%%%%%%%%%%%%%%%%%%%%%%%%%%%%%
\subsection{Unprocessed Newsgroup Posts}
\label{ctbg0:ssre0:sunp0}

Have to process these newsgroup posts:

\begin{tiny}
\begin{verbatim}
I have the need in a microcontroller application to calculate floor(sqrt(x)) 
where x is approximately in the range of 2^32 - 2^64.

What are the algorithms I should look at?

These small microcontrollers are characterized by having a 8-bit times 8-bit 
to give 16-bit result integer unsigned multiply instruction, and a similar 
16/8 division instruction to give a quotient and a remainder. 
Mulitplications of larger integers have to be done by multiplying the bytes 
and adding.

I'm aware of these algorithms:

a)Adding consecutive odd integers and counting how many you have to add, 
i.e. 1 + 3 = 4, 4 + 5 = 9, 9 + 7 = 16, 16 + 9 = 25, etc.

b)Trial squaring, testing setting each bit to "1", i.e.

http://www.uculib.com/vvcuculib01/viewvc.cgi/uculib01/src/stm8/cosmic/modxx/atu8sqrtfrxx/src/atu8sqrtfrxx.c?revision=1.3&view=markup

c)Another method that seems to work (don't know the name of it), here is 
source code:

http://www.uculib.com/vvcuculib01/viewvc.cgi/uculib01/src/stm8/cosmic/modxx/atu16sqrtx10frxx/src/atu16sqrtx10frxx.c?revision=1.3&view=markup

What other options should I investigate?

================================================================================
Here is some C code for integer square roots using many different 
algorithms:
http://cap.connx.com/chess-engines/new-approach/jsqrt0.ZIP

Worth a look:
http://www.azillionmonkeys.com/qed/sqroot.html

Consider also:
http://www.google.com/#hl=en&source=hp&q=fast+integer+square+root&rlz=
1W1GGLD_en&aq=0&aqi=g2&oq=fast+integer+sq&fp=a048890d3c90c6fc
================================================================================
> I have the need in a microcontroller application to calculate 
> floor(sqrt(x)) where x is approximately in the range of 2^32 - 2^64.

x = y^2 - (2^16)^2

> What are the algorithms I should look at?

Google is your friend  :>  There are many sqrt algorithms with
different tradeoffs (speed/size).

I would, instead, ask that you might want to examine what *else*
you know about "x".  I.e., you've already constrained the range
(why?  does that give you any other information that can be
exploited?); are there any other characteristics about how
"x" is synthesized that you can exploit to divide and conquer?

E.g. if x = y * z, then knowing sqrt(y) and/or sqrt(z) simplifies
your problem.

> These small microcontrollers are characterized by having a 8-bit times 
> 8-bit to give 16-bit result integer unsigned multiply instruction, and a 
> similar 16/8 division instruction to give a quotient and a remainder. 
> Mulitplications of larger integers have to be done by multiplying the 
> bytes and adding.
> 
> I'm aware of these algorithms:
> 
> a)Adding consecutive odd integers and counting how many you have to add, 
> i.e. 1 + 3 = 4, 4 + 5 = 9, 9 + 7 = 16, 16 + 9 = 25, etc.
> 
> b)Trial squaring, testing setting each bit to "1", i.e.
> 
> http://www.uculib.com/vvcuculib01/viewvc.cgi/uculib01/src/stm8/cosmic/modxx/atu8sqrtfrxx/src/atu8sqrtfrxx.c?revision=1.3&view=markup 
> 
> c)Another method that seems to work (don't know the name of it), here is 
> source code:
> 
> http://www.uculib.com/vvcuculib01/viewvc.cgi/uculib01/src/stm8/cosmic/modxx/atu16sqrtx10frxx/src/atu16sqrtx10frxx.c?revision=1.3&view=markup 
> 
> What other options should I investigate?
================================================================================
> Hi,
>
> I have the need in a microcontroller application to calculate floor(sqrt(x))
> where x is approximately in the range of 2^32 - 2^64.
>
> What are the algorithms I should look at?
>
> These small microcontrollers are characterized by having a 8-bit times 8-bit
> to give 16-bit result integer unsigned multiply instruction, and a similar
> 16/8 division instruction to give a quotient and a remainder.
> Mulitplications of larger integers have to be done by multiplying the bytes
> and adding.
>
> I'm aware of these algorithms:
>
> a)Adding consecutive odd integers and counting how many you have to add,
> i.e. 1 + 3 = 4, 4 + 5 = 9, 9 + 7 = 16, 16 + 9 = 25, etc.
>
> b)Trial squaring, testing setting each bit to "1", i.e.
>
> http://www.uculib.com/vvcuculib01/viewvc.cgi/uculib01/src/stm8/cosmic...
>
> c)Another method that seems to work (don't know the name of it), here is
> source code:
>
> http://www.uculib.com/vvcuculib01/viewvc.cgi/uculib01/src/stm8/cosmic...
>
> What other options should I investigate?
>
> Thanks, Datesfat

The one I have been using last 15-20 years is probably your B, at
least sounds
like what I made up back then - successive approximation for each bit
of the result.
If multiplication is fast on the core you will be using it it is hard
to beat.

Dimiter
================================================================================
> Hi,
>
> I have the need in a microcontroller application to calculate 
> floor(sqrt(x))
> where x is approximately in the range of 2^32 - 2^64.
>
> What are the algorithms I should look at?
>
> These small microcontrollers are characterized by having a 8-bit times 
> 8-bit
> to give 16-bit result integer unsigned multiply instruction, and a similar
> 16/8 division instruction to give a quotient and a remainder.
> Mulitplications of larger integers have to be done by multiplying the 
> bytes
> and adding.
>
> I'm aware of these algorithms:
>
> a)Adding consecutive odd integers and counting how many you have to add,
> i.e. 1 + 3 = 4, 4 + 5 = 9, 9 + 7 = 16, 16 + 9 = 25, etc.
>
> b)Trial squaring, testing setting each bit to "1", i.e.
>
> http://www.uculib.com/vvcuculib01/viewvc.cgi/uculib01/src/stm8/cosmic...
>
> c)Another method that seems to work (don't know the name of it), here is
> source code:
>
> http://www.uculib.com/vvcuculib01/viewvc.cgi/uculib01/src/stm8/cosmic...
>
> What other options should I investigate?
>
> Thanks, Datesfat
>
>The one I have been using last 15-20 years is probably your B, at
>least sounds
>like what I made up back then - successive approximation for each bit
>of the result.
>If multiplication is fast on the core you will be using it it is hard
>to beat.

Thanks for the insight.

For small operands, I agree with your analysis.

However, for larger operands, the Babylonian method:

http://en.wikipedia.org/wiki/Methods_of_computing_square_roots#Babylonian_method

looks very promising.

The issue is that for the cost of one division (the /2 and addition is 
virtually free) you get far more than one bit of additional information 
about the square root.

For larger operands, I don't believe that (b) will be the right answer any 
longer.

But I'm not sure yet ...
================================================================================
> a)Adding consecutive odd integers and counting how many you have to add, 
> i.e. 1 + 3 = 4, 4 + 5 = 9, 9 + 7 = 16, 16 + 9 = 25, etc.

That quite certainly can't work for the input range you're targeting. 
You would be looking at up to four billion iterations if x was 2^64

> b)Trial squaring, testing setting each bit to "1", i.e.

A whole lot better already.  It becomes even better if you update the 
square of your current result differentially, applying the binomial 
formula.  As you update the result-to-be ('res' in the following), by 
adding a 1-bit at position 'k':

  x --> x + 2^k

it's square, which eventually should be equal to the input number, 
updates too:

x^2 --> x^2 + 2*(2^k)*x + (2^k)^2
               = x^2 + x<<(k+1) + 1<<(2*k)

That's two additions and a bit of shifting.  Even an 8-bit CPU should be 
able to manage that in reasonable time.  Basically you go looking for a 
bit position 'k' such that the current difference between x^2 and the 
target value is still larger than or equal to (x<<(k+1)+1<<(2*k)).

This is actually a special case of an algorithm people used to be taught 
in higher schools.  Just like we all (hopefully) learned to do long 
multiplication and long division in elementary school, there used to be 
an algorithm for long square-roots.  I've seen it in a textbook used to 
teach metal workers' apprentices from the 1940s.  It's a little daunting 
in decimal numbers on paper, but boils down to the above when performed 
in binary: you figure out one digit per iteration by looking at the 
difference between the target figure and square of the result-to-be. 
It's also somewhat similar to long division in that way.

> What other options should I investigate?

One traditional trick for sqrt(), and some other inverse functions, too, 
  is to treat it as a root-finding problem for the function

f(x) := x^2 - y

The x that solves f(x)=0 is sqrt(y).  You can solve that using the 
Newton-Raphson iteration algorithm, a.k.a. the tangent method:

x' = x - f(x)/f'(x)
   = x - (x^2 - y) / (2 * x)
   = x - x/2 - y/(2*x)
   = x/2 - (y/2)/x

This requires long division, but pays off by converging on the solution 
real fast, once you've come anywhere near it.

This algorithm and the bit-by-bit one above are somewhat related.  The 
search for the right bit position, 'k', is effectively a simplified 
version of the division (y/2)/x.
================================================================================
> I have the need in a microcontroller application to calculate floor(sqrt(x)) 
> where x is approximately in the range of 2^32 - 2^64.
> 
> What are the algorithms I should look at?

Newton-Raphson or bisection, depending upon how slow division is. N-R is
asymptotically faster (the number of correct digits doubles at each step),
but each step requires a division.

Bisection is essentially your option b), but it can be computed much more
efficiently than the algorithm given. You don't need arbitrary multiplies,
only shifts, as:

(a+2^k)^2 = a^2 + 2.a.2^k + (2^k)^2
          = a^2 + a.2^(k+1) + 2^2k

IOW:

uint32_t isqrt(uint64_t x)
{
    uint64_t a = 0;
    uint64_t a2 = 0;
    int k;

    for (k = 31; k >= 0; k--) {

uint64_t t = a2 + (a << (k+1)) + ((uint64_t)1 << (k + k));
if (x > t) {
    a += 1 << k;
    a2 = t;
}
    }

    return a;
}

Depending upon the CPU, it may be faster to calculate the a.2^(k+1) and
2^2k terms incrementally.
================================================================================
Wiki

http://en.wikipedia.org/wiki/Methods_of_computing_square_roots\\end{verbatim}
================================================================================
Noticed your post on sci.math. I am surprised no one mentioned this one
from Joe Leherbauer that only uses shifts and adds. I cannot find the
original usenet message but a copy is here:

http://groups.google.com/group/comp.lang.c/browse\_thread/thread/14ed11f99b6ac8ab/d4297a8de3ead321?hl=en\&ie=UTF-8\&q=\%22joe+Leherbauer\%22

I clipped the message:

<----------------------clipped from message---------------------------->
The function below is probably the fastest you can get with integer
operations only.  It executes in constant time of 16 loop iterations
assuming a long is 32 bits, i.e. it's O(log(n)).
As a by-product it also produces the square root remainder.
More formally, it computes s and r, such that:

s = floor(sqrt(n))
r = n - s * s

I have to admit that I did not invent this.
It is from an anonymous Usenet source.

unsigned long
i\_sqrt (unsigned long n, unsigned long * rem)
{
 unsigned long r, s, t;

 r = 0;
 t = (~0UL >> 2) + 1;   /* largest power of 4 supported by data type */
 do
   {
     s = r + t;
     if ( n >= s )
       {
         n -= s;
         r = s + t;
       }
     r >>= 1;
   }
 while ( (t >>= 2) != 0 );
 *rem = n;

 return r;

}

---
Joe Leherbauer             Leherbauer at telering dot at
                          Registered Linux User \# 215803
================================================================================
\end{tiny}


%%%%%%%%%%%%%%%%%%%%%%%%%%%%%%%%%%%%%%%%%%%%%%%%%%%%%%%%%%%%%%%%%%%%%%%%%%%%%%%
%%%%%%%%%%%%%%%%%%%%%%%%%%%%%%%%%%%%%%%%%%%%%%%%%%%%%%%%%%%%%%%%%%%%%%%%%%%%%%%
%%%%%%%%%%%%%%%%%%%%%%%%%%%%%%%%%%%%%%%%%%%%%%%%%%%%%%%%%%%%%%%%%%%%%%%%%%%%%%%
\section[\emph{UcuAtS32S16v2CpDiva2FRxx(\protect\mbox{\protect$\cdot$})} Function Calculation Method]
        {\emph{UcuAtS32S16v2CpDiva2FRxx(\protect\mbox{\protect\boldmath $\cdot$})} Function Calculation Method}
\label{ctbg0:svec2}

\index{UcuAtS32S16v2CpDiva2FRxx()@\emph{UcuAtS32S16v2CpDiva2FRxx($\cdot$)}}%
The nominal calculation performed by the
\emph{UcuAtS32S16v2CpDiva2FRxx($\cdot$)} function is

\begin{equation}
\label{eq:ctbg0:svec2:01}
g(a_x, a_y, b_x, b_y)
=
\left\lfloor \frac{\vec{a} \times \vec{b}}{| \vec{b} | \hat{k}} \right\rfloor
=
\left\lfloor \frac{a_x b_y - a_y b_x}{\sqrt{b_x^2 + b_y^2}} \right\rfloor ,
\end{equation}

\noindent{}where the floor function of a negative argument rounds towards zero.

Clearly, (\ref{eq:ctbg0:svec2:01}) isn't a convenient method of calcuation for an
inexpensive microcontroller, as the result
of the square root calculation might need to include a fractional part.

(\ref{eq:ctbg0:svec2:01}) can be modified by squaring then taking the square root,
making adjustments for the loss of sign information caused by squaring:

\begin{equation}
\label{eq:ctbg0:svec2:02}
\left\lfloor \frac{a_x b_y - a_y b_x}{\sqrt{b_x^2 + b_y^2}} \right\rfloor
=
\left\lfloor \sqrt{\frac{(a_x b_y - a_y b_x)^2}{b_x^2 + b_y^2}} \right\rfloor sgn (a_x b_y - a_y b_x).
\end{equation}

Note in (\ref{eq:ctbg0:svec2:02})
the the remainder of the division can be discarded, i.e.

\begin{equation}
\label{eq:ctbg0:svec2:03}
\left\lfloor \sqrt{\frac{(a_x b_y - a_y b_x)^2}{b_x^2 + b_y^2}} \right\rfloor
=
\left\lfloor \sqrt{\left\lfloor\frac{(a_x b_y - a_y b_x)^2}{b_x^2 + b_y^2}\right\rfloor} \right\rfloor .
\end{equation}

\noindent{}(\ref{eq:ctbg0:svec2:03}) is valid because only the square root of an integer can be 
an integer, so flooring the input to a square root calculation whose output is floored
will not have an effect.

The size of the integer to be squared in (\ref{eq:ctbg0:svec2:03}) must be established.
$a_x b_y - a_y b_x$ may have a magnitude as large as $2^{31}-2^{15}$, so 31 bits are required (for an unsigned representation).
In practice, 32 bits (4 bytes) will be used for storage.

The term $(a_x b_y - a_y b_x)^2$ may be as large as $(2^{31}-2^{15})^2 = 2^{62}-2^{47}+2^{30}$, so 62 bits are required.
In practice, 64 bits (8 bytes) will be used for storage.

The most difficult question to answer is the upper bound on the result of the
division.  One would suspect from the form of (\ref{eq:ctbg0:svec2:03})
that there is an upper bound less than $2^{62}-2^{47}+2^{30}$, as the
numerator and denominator both increase with increasing $b_x$ or $b_y$.
If an upper bound smaller than $2^{62}-2^{47}+2^{30}$ exists, it would make
both the division and the square root extraction more efficient.

An upper bound on the result of the division in (\ref{eq:ctbg0:svec2:03})
need not be overly tight.  The upper bound is only to assist in devising
efficient calculation.

The maximum magnitude of $a_x b_y - a_y b_x$ can only occur when a negative
number is subtracted from a positive, or a positive number is subtracted from
a negative.  The most positive number that can be formed as the product
of two UCU\_SINT16's is $2^{15}2^{15} = 2^{30}$.  The most negative number that can be formed
is $-2^{15}(2^{15}-1) = -2^{30}+2^{15}$.  The maximum magnitude of
$a_x b_y - a_y b_x$ is thus $2^{30} - (-2^{30}+2^{15}) = 2^{31} - 2^{15}$.

To simplify the algebra (in exchange for a looser bound),
one can assume that both positive and negative
UCU\_SINT16's may reach a magnitude of $2^{15}$.  Assuming
that $a_x = a_y = 2^{15}$, the maximum value of
(\ref{eq:ctbg0:svec2:03}) is

\begin{eqnarray}
\nonumber
& 
  \displaystyle{
  \sqrt{\frac{(2^{15} b_x + 2^{15} b_y)^2}{b_x^2 + b_y^2}}
  =
  \sqrt{\frac{2^{30}(b_x^2 + 2 b_x b_y + b_y^2)}{b_x^2 + b_y^2}}
  }
& \\
\label{eq:ctbg0:svec2:04}
& 
  \displaystyle{
  = \sqrt{2^{30} \frac{b_x^2 + b_y^2}{b_x^2 + b_y^2} + 2^{31} \frac{b_x b_y}{b_x^2 + b_y^2}}
  }
& \\
\nonumber
& 
  \displaystyle{
  = \sqrt{2^{30} + 2^{31} \frac{b_x b_y}{b_x^2 + b_y^2}}
  }
&
\end{eqnarray}

(\ref{eq:ctbg0:svec2:04}) implies that the maximum value of (\ref{eq:ctbg0:svec2:03})
depends on the maximum of the function

\begin{equation}
\label{eq:ctbg0:svec2:05}
f(b_x, b_y) = \frac{b_x b_y}{b_x^2 + b_y^2} , \;\; b_x, b_y \in \mathbb{Z}^+ .
\end{equation}

Several helpful posters\footnote{Thanks to Ray Vickson, Gerry Myerson, Rob Johnson, and ``Scattered''.} 
on \texttt{sci.math} provided proofs that
(\ref{eq:ctbg0:svec2:05}) could not exceed 1/2.  The proof that I found easiest to understand 
was provided by Gerry Myerson.  Consider the function

\begin{equation}
\label{eq:ctbg0:svec2:06}
h(b_x, b_y)
= 
\frac{1}{2} - f(b_x, b_y)
= 
\frac{1}{2} \left( \frac{b_x^2 - 2 b_x b_y + b_y^2}{b_x^2 + b_y^2} \right)
= 
\frac{1}{2} \left( \frac{(b_x - b_y)^2}{b_x^2 + b_y^2} \right).
\end{equation}

$h(b_x, b_y)$ is non-negative over the domain of interest, proving that $f(b_x, b_y)$ cannot exceed $1/2$.

Substituting the maximum value of 1/2 into (\ref{eq:ctbg0:svec2:04})
leads to

\begin{equation}
\label{eq:ctbg0:svec2:07}
\frac{(a_x b_y - a_y b_x)^2}{b_x^2 + b_y^2} 
\leq 
2^{30} + 2^{31} \left( \frac{1}{2} \right)
=
2^{30} + 2^{30}
=
2^{31} .
\end{equation}

\noindent{}(\ref{eq:ctbg0:svec2:07})
establishes that the result of the division may not exceed 32 bits.  This is
an important result, as the dividend of the division is 64 bits, requiring
64 rounds in the general case to obtain a quotient.  Knowledge that the
quotient may not exceed 32 bits will approximately halve the division time, as
only 32 iterations rather than 64 need to be performed.

The square root extraction will involve a 32-bit input (the quotient from the
division) and a 16-bit result.


%%%%%%%%%%%%%%%%%%%%%%%%%%%%%%%%%%%%%%%%%%%%%%%%%%%%%%%%%%%%%%%%%%%%%%%%%%
\noindent\begin{figure}[!b]
\noindent\rule[-0.25in]{\textwidth}{1pt}
\begin{tiny}
\begin{verbatim}
$RCSfile: c_tbg0.tex,v $
$Source: /home/dashley/cvsrep/uculib01/uculib01/doc/manual/c_tbg0/c_tbg0.tex,v $
$Revision: 1.12 $
$Author: dashley $
$Date: 2010/04/05 15:13:50 $
\end{verbatim}
\end{tiny}
\noindent\rule[0.25in]{\textwidth}{1pt}
\end{figure}

%%%%%%%%%%%%%%%%%%%%%%%%%%%%%%%%%%%%%%%%%%%%%%%%%%%%%%%%%%%%%%%%%%%%%%%%%%%%%%%
%$Log: c_tbg0.tex,v $
%Revision 1.12  2010/04/05 15:13:50  dashley
%Edits.
%
%Revision 1.11  2010/04/05 12:17:12  dashley
%Edits.
%
%Revision 1.10  2010/04/01 21:07:34  dashley
%Edits.
%
%Revision 1.9  2010/04/01 20:21:50  dashley
%Edits.
%
%Revision 1.8  2010/03/03 16:30:35  dashley
%Edits.
%
%Revision 1.7  2010/03/03 00:19:21  dashley
%Edits.
%
%Revision 1.6  2010/02/05 03:04:01  dashley
%Additional square root information added.
%
%Revision 1.5  2010/02/05 02:58:12  dashley
%Newsgroup post text added.
%
%Revision 1.4  2010/01/28 21:18:33  dashley
%a)Chapter start quotes removed.
%b)Aesthetic comment line added at the bottom of most files.
%
%Revision 1.3  2007/10/08 18:16:34  dtashley
%Edits.
%
%Revision 1.2  2007/10/07 18:11:22  dtashley
%Edits.
%
%Revision 1.1  2007/10/07 01:22:01  dtashley
%Initial checkin.
%
%End of $RCSfile: c_tbg0.tex,v $.
%%%%%%%%%%%%%%%%%%%%%%%%%%%%%%%%%%%%%%%%%%%%%%%%%%%%%%%%%%%%%%%%%%%%%%%%%%%%%%%

