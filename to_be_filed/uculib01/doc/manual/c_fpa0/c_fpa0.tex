%$Header: /home/dashley/cvsrep/uculib01/uculib01/doc/manual/c_fpa0/c_fpa0.tex,v 1.4 2010/02/22 15:40:28 dashley Exp $

\chapter{Fixed-Point Arithmetic Functions}

\label{cfpa0}

%%%%%%%%%%%%%%%%%%%%%%%%%%%%%%%%%%%%%%%%%%%%%%%%%%%%%%%%%%%%%%%%%%%%%%%%%%%%%%%
%%%%%%%%%%%%%%%%%%%%%%%%%%%%%%%%%%%%%%%%%%%%%%%%%%%%%%%%%%%%%%%%%%%%%%%%%%%%%%%
%%%%%%%%%%%%%%%%%%%%%%%%%%%%%%%%%%%%%%%%%%%%%%%%%%%%%%%%%%%%%%%%%%%%%%%%%%%%%%%
\section{Introduction and Overview}
\label{cfpa0:siov0}

TBD.


%%%%%%%%%%%%%%%%%%%%%%%%%%%%%%%%%%%%%%%%%%%%%%%%%%%%%%%%%%%%%%%%%%%%%%%%%%%%%%%
%%%%%%%%%%%%%%%%%%%%%%%%%%%%%%%%%%%%%%%%%%%%%%%%%%%%%%%%%%%%%%%%%%%%%%%%%%%%%%%
%%%%%%%%%%%%%%%%%%%%%%%%%%%%%%%%%%%%%%%%%%%%%%%%%%%%%%%%%%%%%%%%%%%%%%%%%%%%%%%
\section{UCU\_UINT16 [0, 1024] [0,1] Bounded Functions}
\label{cfpa0:subf0}


%%%%%%%%%%%%%%%%%%%%%%%%%%%%%%%%%%%%%%%%%%%%%%%%%%%%%%%%%%%%%%%%%%%%%%%%%%%%%%%
%%%%%%%%%%%%%%%%%%%%%%%%%%%%%%%%%%%%%%%%%%%%%%%%%%%%%%%%%%%%%%%%%%%%%%%%%%%%%%%
%%%%%%%%%%%%%%%%%%%%%%%%%%%%%%%%%%%%%%%%%%%%%%%%%%%%%%%%%%%%%%%%%%%%%%%%%%%%%%%
\subsection[\emph{UcuFpUm1024uProjIntoRU16aRxx(\protect\mbox{\protect$\cdot$})}]
           {\emph{UcuFpUm1024uProjIntoRU16aRxx(\protect\mbox{\protect\boldmath $\cdot$})}}
%Subsection tag:  faa1
\label{cfpa0:subf0:sfaa0}

\index{UcuFpUm1024uProjIntoRU16aRxx()@\emph{UcuFpUm1024uProjIntoRU16aRxx($\cdot$)}}%

\noindent\textbf{PROTOTYPE}
\begin {list}{}{\setlength{\leftmargin}{0.25in}\setlength{\topsep}{0.0in}}
\item
\begin{verbatim}
UCU_UINT16 UcuFpUm1024uProjIntoRU16aRxx(
                                       UCU_UINT16 x, 
                                       UCU_UINT16 x_max, 
                                       )
\end{verbatim}
\end{list}
\vspace{2.8ex}

\noindent\textbf{SYNOPSIS}
\begin{list}{}{\setlength{\leftmargin}{0.25in}\setlength{\topsep}{0.0in}}
\item Linearly scales from $[0, x_{max}]$ to $[0, 1024]$, rounding the result
      to the nearest integer with a downward bias.  The result calculated is
      $\displaystyle{\left\lfloor \frac{1024 x + \left\lfloor \displaystyle{\frac{x_{max} - 1}{2}} \right\rfloor}{x_{max}} \right\rfloor}$,
      with the provision that the function output will never exceed 1024.
\end{list}
\vspace{2.8ex}

\noindent\textbf{INPUTS}
\begin{list}{}{\setlength{\leftmargin}{0.5in}\setlength{\itemindent}{-0.25in}\setlength{\topsep}{0.0in}\setlength{\partopsep}{0.0in}}
\item \emph{\textbf{x}}\\
      The input to scale.
\item \emph{\textbf{x\_max}}\\
      The value of $x$ that should correspond to 1024.
\end{list}
\vspace{2.8ex}

\noindent\textbf{OUTPUT}
\begin{list}{}{\setlength{\leftmargin}{0.25in}\setlength{\topsep}{0.0in}}
\item $\displaystyle{\left\lfloor \frac{1024 x + \left\lfloor \displaystyle{\frac{x_{max} - 1}{2}} \right\rfloor}{x_{max}} \right\rfloor}$,
      with a maximum of 1024.
\end{list}
\vspace{2.8ex}

\noindent\textbf{EXCEPTION CASES}
\begin{list}{}{\setlength{\leftmargin}{0.25in}\setlength{\topsep}{0.0in}}
\item If $x>x_{max}$, 1024 is returned.
\item If $x_{max}=0$, 0 is returned.  (In this
      case, it isn't possible to determine the intent of
      the caller, so 0 is the safest return value.)
\end{list}
\vspace{2.8ex}

\noindent\textbf{INTERRUPT COMPATIBILITY}
\begin{list}{}{\setlength{\leftmargin}{0.25in}\setlength{\topsep}{0.0in}}
\item This function may be used from both non-ISR and ISR software.
\item This function is thread-safe.
\end{list}
\vspace{2.8ex}

\noindent\textbf{EXECUTION TIME}
\begin{list}{}{\setlength{\leftmargin}{0.25in}\setlength{\topsep}{0.0in}}
\item TBD.
\end{list}
\vspace{2.8ex}

\noindent\textbf{FUNCTION NAME MNEMONIC}
\begin{list}{}{\setlength{\leftmargin}{0.25in}\setlength{\topsep}{0.0in}}
\item \emph{Um1024u}: unsigned maximum-1024 unsigned.
      \emph{Proj}:    projection.
      \emph{Into}:    into.
      \emph{R}:       result rounded to nearest integer.
      \emph{U16a}:    the argument is UCU\_UINT16.
\end{list}


%%%%%%%%%%%%%%%%%%%%%%%%%%%%%%%%%%%%%%%%%%%%%%%%%%%%%%%%%%%%%%%%%%%%%%%%%%%%%%%
%%%%%%%%%%%%%%%%%%%%%%%%%%%%%%%%%%%%%%%%%%%%%%%%%%%%%%%%%%%%%%%%%%%%%%%%%%%%%%%
%%%%%%%%%%%%%%%%%%%%%%%%%%%%%%%%%%%%%%%%%%%%%%%%%%%%%%%%%%%%%%%%%%%%%%%%%%%%%%%
\subsection[\emph{UcuFpUm1024uMulRRxx(\protect\mbox{\protect$\cdot$})}]
           {\emph{UcuFpUm1024uMulRRxx(\protect\mbox{\protect\boldmath $\cdot$})}}
\label{cfpa0:subf0:sfaa1}

\index{UcuFpUm1024uMulRRxx()@\emph{UcuFpUm1024uMulRRxx($\cdot$)}}%

\noindent\textbf{PROTOTYPE}
\begin {list}{}{\setlength{\leftmargin}{0.25in}\setlength{\topsep}{0.0in}}
\item
\begin{verbatim}
UCU_UINT16 UcuFpUm1024uMulRRxx(
                              UCU_UINT16 x_1, 
                              UCU_UINT16 x_2, 
                              )
\end{verbatim}
\end{list}
\vspace{2.8ex}

\noindent\textbf{SYNOPSIS}
\begin{list}{}{\setlength{\leftmargin}{0.25in}\setlength{\topsep}{0.0in}}
\item Calculates $x_1 x_2$, treating $x_1$ and $x_2$ as fixed-point numbers with 1024 corresponding
      to 1, and with $x_1$ and $x_2$ having a maximum of 1. The output is rounded to the nearest
      integer with a slight downward bias.
\item The precise function implemented is
      \begin{equation}
      \left\lfloor \frac{min(x_1, 1024) \; min(x_2, 1024) + 511}{1024} \right\rfloor.
      \end{equation}
\item The function output cannot exceed 1024.
\end{list}
\vspace{2.8ex}

\noindent\textbf{INPUTS}
\begin{list}{}{\setlength{\leftmargin}{0.5in}\setlength{\itemindent}{-0.25in}\setlength{\topsep}{0.0in}\setlength{\partopsep}{0.0in}}
\item \emph{\textbf{x\_1}}, \emph{\textbf{x\_2}} \\
      The inputs to multiply together.
\end{list}
\vspace{2.8ex}

\noindent\textbf{OUTPUT}
\vspace{0.4ex}
\begin{list}{}{\setlength{\leftmargin}{0.25in}\setlength{\topsep}{0.0in}}
\item  $\displaystyle{\left\lfloor \frac{min(x_1, 1024) \; min(x_2, 1024) + 511}{1024} \right\rfloor}$.
\end{list}
\vspace{2.8ex}

\noindent\textbf{EXCEPTION CASES}
\begin{list}{}{\setlength{\leftmargin}{0.25in}\setlength{\topsep}{0.0in}}
\item As indicated above, $x_1$ and $x_2$ are clipped at 1024 before being used
      in the calculation.
\end{list}
\vspace{2.8ex}

\noindent\textbf{INTERRUPT COMPATIBILITY}
\begin{list}{}{\setlength{\leftmargin}{0.25in}\setlength{\topsep}{0.0in}}
\item This function may be used from both non-ISR and ISR software.
\item This function is thread-safe.
\end{list}
\vspace{2.8ex}

\noindent\textbf{EXECUTION TIME}
\begin{list}{}{\setlength{\leftmargin}{0.25in}\setlength{\topsep}{0.0in}}
\item TBD.
\end{list}
\vspace{2.8ex}

\noindent\textbf{FUNCTION NAME MNEMONIC}
\begin{list}{}{\setlength{\leftmargin}{0.25in}\setlength{\topsep}{0.0in}}
\item \emph{Um1024u}: unsigned maximum-1024 unsigned.
      \emph{Mul}:     multiplication.
      \emph{R}:       result is rounded to nearest integer (with slight downward bias).
\end{list}


%%%%%%%%%%%%%%%%%%%%%%%%%%%%%%%%%%%%%%%%%%%%%%%%%%%%%%%%%%%%%%%%%%%%%%%%%%
\noindent\begin{figure}[!b]
\noindent\rule[-0.25in]{\textwidth}{1pt}
\begin{tiny}
\begin{verbatim}
$RCSfile: c_fpa0.tex,v $
$Source: /home/dashley/cvsrep/uculib01/uculib01/doc/manual/c_fpa0/c_fpa0.tex,v $
$Revision: 1.4 $
$Author: dashley $
$Date: 2010/02/22 15:40:28 $
\end{verbatim}
\end{tiny}
\noindent\rule[0.25in]{\textwidth}{1pt}
\end{figure}

%%%%%%%%%%%%%%%%%%%%%%%%%%%%%%%%%%%%%%%%%%%%%%%%%%%%%%%%%%%%%%%%%%%%%%%%%%%%%%%
%$Log: c_fpa0.tex,v $
%Revision 1.4  2010/02/22 15:40:28  dashley
%Function added.
%
%Revision 1.3  2010/02/22 00:16:02  dashley
%Documentation added.
%
%Revision 1.2  2010/01/28 21:18:32  dashley
%a)Chapter start quotes removed.
%b)Aesthetic comment line added at the bottom of most files.
%
%Revision 1.1  2010/01/24 03:00:41  dashley
%Initial checkin.
%
%End of $RCSfile: c_fpa0.tex,v $.
%%%%%%%%%%%%%%%%%%%%%%%%%%%%%%%%%%%%%%%%%%%%%%%%%%%%%%%%%%%%%%%%%%%%%%%%%%%%%%%

