%$Header: /home/dashley/cvsrep/uculib01/uculib01/doc/manual/c_ami0/c_ami0.tex,v 1.12 2010/02/24 14:50:36 dashley Exp $

\chapter{Array Manipulation Functions}

\label{cami0}


%%%%%%%%%%%%%%%%%%%%%%%%%%%%%%%%%%%%%%%%%%%%%%%%%%%%%%%%%%%%%%%%%%%%%%%%%%%%%%%
%%%%%%%%%%%%%%%%%%%%%%%%%%%%%%%%%%%%%%%%%%%%%%%%%%%%%%%%%%%%%%%%%%%%%%%%%%%%%%%
%%%%%%%%%%%%%%%%%%%%%%%%%%%%%%%%%%%%%%%%%%%%%%%%%%%%%%%%%%%%%%%%%%%%%%%%%%%%%%%
\section{Introduction and Overview}
%Section tag:  ami0
\label{cami0:siov0}


%%%%%%%%%%%%%%%%%%%%%%%%%%%%%%%%%%%%%%%%%%%%%%%%%%%%%%%%%%%%%%%%%%%%%%%%%%%%%%%
%%%%%%%%%%%%%%%%%%%%%%%%%%%%%%%%%%%%%%%%%%%%%%%%%%%%%%%%%%%%%%%%%%%%%%%%%%%%%%%
%%%%%%%%%%%%%%%%%%%%%%%%%%%%%%%%%%%%%%%%%%%%%%%%%%%%%%%%%%%%%%%%%%%%%%%%%%%%%%%
\section{Array Element Decrement Not Below Zero Functions}
%Section tag:  adz0
\label{cami0:sdez0}


%%%%%%%%%%%%%%%%%%%%%%%%%%%%%%%%%%%%%%%%%%%%%%%%%%%%%%%%%%%%%%%%%%%%%%%%%%%%%%%
%%%%%%%%%%%%%%%%%%%%%%%%%%%%%%%%%%%%%%%%%%%%%%%%%%%%%%%%%%%%%%%%%%%%%%%%%%%%%%%
%%%%%%%%%%%%%%%%%%%%%%%%%%%%%%%%%%%%%%%%%%%%%%%%%%%%%%%%%%%%%%%%%%%%%%%%%%%%%%%
\subsection[\emph{UcuAmU8aDnbzU16nRxn(\protect\mbox{\protect$\cdot$})}]
           {\emph{UcuAmU8aDnbzU16nRxn(\protect\mbox{\protect\boldmath $\cdot$})}}
%Section tag:  dez0
\label{cami0:sadx0:sdez0}

\index{UcuAmU8aDnbzU16nRxn()@\emph{UcuAmU8aDnbzU16nRxn($\cdot$)}}%

\noindent\textbf{PROTOTYPE}
\begin {list}{}{\setlength{\leftmargin}{0.25in}\setlength{\topsep}{0.0in}}
\item
\begin{verbatim}
void UcuAmU8aDnbzU16nRxn(
                        UCU_UINT8  *a,
                        UCU_UINT16  n
                        )
\end{verbatim}
\end{list}
\vspace{2.8ex}

\noindent\textbf{SYNOPSIS}
\begin{list}{}{\setlength{\leftmargin}{0.25in}\setlength{\topsep}{0.0in}}
\item Decrements each element of an array of UCU\_UINT8, but not below zero.
\end{list}
\vspace{2.8ex}

\noindent\textbf{INPUTS}
\begin{list}{}{\setlength{\leftmargin}{0.5in}\setlength{\itemindent}{-0.25in}\setlength{\topsep}{0.0in}\setlength{\partopsep}{0.0in}}
\item \emph{\textbf{a}}\\
      Pointer to the first element of the array.  This pointer will only
      be dereferenced if $n>0$.
\item \emph{\textbf{n}}\\
      The number of elements to be decremented.

      If $n=0$, \texttt{a} will not be
      dereferenced and no elements of the array will be modified.

      If $n>0$, \texttt{a}
      will be dereferenced and must be a valid pointer.
\end{list}
\vspace{2.8ex}

\noindent\textbf{OUTPUTS}
\begin{list}{}{\setlength{\leftmargin}{0.25in}\setlength{\topsep}{0.0in}}
\item None.
\end{list}
\vspace{2.8ex}

\noindent\textbf{INTERRUPT COMPATIBILITY}
\begin{list}{}{\setlength{\leftmargin}{0.25in}\setlength{\topsep}{0.0in}}
\item This function may be called from non-ISR or ISR software.
\item This function does not protect array accesses in any way, and so
      is not safe to use to operate on data shared with another thread.
      At least two types of multi-threaded shared data defects may occur if
      this function is used on data shared between threads:
      \begin{itemize}
      \item Race conditions may occur when accessing a single array element.
            The test and decrement are not guaranteed to be atomic, and
            the decrement is not guaranteed to be atomic.
      \item The operation performed on the entire array is not guaranteed
            to be atomics.
      \end{itemize}
\end{list}
\vspace{2.8ex}

\noindent\textbf{EXECUTION TIME}
\begin{list}{}{\setlength{\leftmargin}{0.25in}\setlength{\topsep}{0.0in}}
\item TBD.
\end{list}
\vspace{2.8ex}

\noindent\textbf{FUNCTION NAME MNEMONIC}
\begin{list}{}{\setlength{\leftmargin}{0.25in}\setlength{\topsep}{0.0in}}
\item \emph{U8a}:   operates on an array of UCU\_UINT8.
      \emph{Dnbz}:  decrement, but not below zero.
      \emph{U16n}:  the number of elements to operate on is specified
                    as a UCU\_UINT16.
\end{list}


%%%%%%%%%%%%%%%%%%%%%%%%%%%%%%%%%%%%%%%%%%%%%%%%%%%%%%%%%%%%%%%%%%%%%%%%%%%%%%%
%%%%%%%%%%%%%%%%%%%%%%%%%%%%%%%%%%%%%%%%%%%%%%%%%%%%%%%%%%%%%%%%%%%%%%%%%%%%%%%
%%%%%%%%%%%%%%%%%%%%%%%%%%%%%%%%%%%%%%%%%%%%%%%%%%%%%%%%%%%%%%%%%%%%%%%%%%%%%%%
\subsection[\emph{UcuAmU16aDnbzU16nRxn(\protect\mbox{\protect$\cdot$})}]
           {\emph{UcuAmU16aDnbzU16nRxn(\protect\mbox{\protect\boldmath $\cdot$})}}
\label{cami0:sadx0:sdez1}

\index{UcuAmU16aDnbzU16nRxn()@\emph{UcuAmU16aDnbzU16nRxn($\cdot$)}}%

\noindent\textbf{PROTOTYPE}
\begin {list}{}{\setlength{\leftmargin}{0.25in}\setlength{\topsep}{0.0in}}
\item
\begin{verbatim}
void UcuAmU16aDnbzU16nRxn(
                         UCU_UINT16 *in_arg,
                         UCU_UINT16  in_nelem
                         )
\end{verbatim}
\end{list}
\vspace{2.8ex}

\noindent\textbf{SYNOPSIS}
\begin{list}{}{\setlength{\leftmargin}{0.25in}\setlength{\topsep}{0.0in}}
\item The \emph{UcuAmU16aDnbzU16nRxn($\cdot$)} function is identical in every
      respect to the \emph{UcuAmU8aDnbzU16nRxn($\cdot$)} function
      (\S{}\ref{cami0:sadx0:sdez0}, p. \pageref{cami0:sadx0:sdez0})
      except that each element of the array is 16 bits
      rather than 8.
\end{list}
\vspace{2.8ex}


%%%%%%%%%%%%%%%%%%%%%%%%%%%%%%%%%%%%%%%%%%%%%%%%%%%%%%%%%%%%%%%%%%%%%%%%%%%%%%%
%%%%%%%%%%%%%%%%%%%%%%%%%%%%%%%%%%%%%%%%%%%%%%%%%%%%%%%%%%%%%%%%%%%%%%%%%%%%%%%
%%%%%%%%%%%%%%%%%%%%%%%%%%%%%%%%%%%%%%%%%%%%%%%%%%%%%%%%%%%%%%%%%%%%%%%%%%%%%%%
\subsection[\emph{UcuAmU32aDnbzU16nRxn(\protect\mbox{\protect$\cdot$})}]
           {\emph{UcuAmU32aDnbzU16nRxn(\protect\mbox{\protect\boldmath $\cdot$})}}
%Section tag:  dez2
\label{cami0:sadx0:sdez2}

\index{UcuAmU32aDnbzU16nRxn()@\emph{UcuAmU32aDnbzU16nRxn($\cdot$)}}%

\noindent\textbf{PROTOTYPE}
\begin {list}{}{\setlength{\leftmargin}{0.25in}\setlength{\topsep}{0.0in}}
\item
\begin{verbatim}
void UcuAmU32aDnbzU16nRxn(
                         UCU_UINT32 *in_arg,
                         UCU_UINT16  in_nelem
                         )
\end{verbatim}
\end{list}
\vspace{2.8ex}

\noindent\textbf{SYNOPSIS}
\begin{list}{}{\setlength{\leftmargin}{0.25in}\setlength{\topsep}{0.0in}}
\item The \emph{UcuAmU32aDnbzU16nRxn($\cdot$)} function is identical in every
      respect to the \emph{UcuAmU8aDnbzU16nRxn($\cdot$)} function
      (\S{}\ref{cami0:sadx0:sdez0}, p. \pageref{cami0:sadx0:sdez0})
      except that each element of the array is 32 bits
      rather than 8.
\end{list}
\vspace{2.8ex}


%%%%%%%%%%%%%%%%%%%%%%%%%%%%%%%%%%%%%%%%%%%%%%%%%%%%%%%%%%%%%%%%%%%%%%%%%%%%%%%
%%%%%%%%%%%%%%%%%%%%%%%%%%%%%%%%%%%%%%%%%%%%%%%%%%%%%%%%%%%%%%%%%%%%%%%%%%%%%%%
%%%%%%%%%%%%%%%%%%%%%%%%%%%%%%%%%%%%%%%%%%%%%%%%%%%%%%%%%%%%%%%%%%%%%%%%%%%%%%%
\section{Right Shift Functions}
%Section tag:  rsf0
\label{cami0:srsf0}


%%%%%%%%%%%%%%%%%%%%%%%%%%%%%%%%%%%%%%%%%%%%%%%%%%%%%%%%%%%%%%%%%%%%%%%%%%%%%%%
%%%%%%%%%%%%%%%%%%%%%%%%%%%%%%%%%%%%%%%%%%%%%%%%%%%%%%%%%%%%%%%%%%%%%%%%%%%%%%%
%%%%%%%%%%%%%%%%%%%%%%%%%%%%%%%%%%%%%%%%%%%%%%%%%%%%%%%%%%%%%%%%%%%%%%%%%%%%%%%
\subsection[\emph{UcuAmU8aRs1U16nRxn(\protect\mbox{\protect$\cdot$})}]
           {\emph{UcuAmU8aRs1U16nRxn(\protect\mbox{\protect\boldmath $\cdot$})}}
%Section tag:  rsh0
\label{cami0:srsf0:srsh0}

\index{UcuAmU8aRs1U16nRxn()@\emph{UcuAmU8aRs1U16nRxn($\cdot$)}}%

\noindent\textbf{PROTOTYPE}
\begin {list}{}{\setlength{\leftmargin}{0.25in}\setlength{\topsep}{0.0in}}
\item
\begin{verbatim}
void UcuAmU8aRs1U16nRxn(
                       UCU_UINT8  *in_arr, 
                       UCU_UINT16  in_nelem, 
                       UCU_UINT8   in_pushval
                       )
\end{verbatim}
\end{list}
\vspace{2.8ex}

\noindent\textbf{SYNOPSIS}
\begin{list}{}{\setlength{\leftmargin}{0.25in}\setlength{\topsep}{0.0in}}
\item Shifts an array of unsigned bytes one position to the right
      (towards larger-numbered array subscripts) and places
      a supplied byte into the leftmost (element [0]) position.
\end{list}
\vspace{2.8ex}

\noindent\textbf{INPUTS}
\begin{list}{}{\setlength{\leftmargin}{0.5in}\setlength{\itemindent}{-0.25in}\setlength{\topsep}{0.0in}\setlength{\partopsep}{0.0in}}
\item \emph{\textbf{in\_arr}}\\
      Pointer to the first element of the array to be shifted.  This pointer will be
      dereferenced only if $in\_nelem > 0$.
\item \emph{\textbf{in\_nelem}}\\
      The total number of elements in the array.  This is frequently
      calculated as \texttt{sizeof(arr) / sizeof(arr[0])}.  (Note that this is
      the \emph{number of elements} in the array, not the size of the array.)
\item \emph{\textbf{in\_pushval}}\\
      The value to be assigned to array element [0].
\end{list}
\vspace{2.8ex}

\noindent\textbf{OUTPUTS}
\begin{list}{}{\setlength{\leftmargin}{0.25in}\setlength{\topsep}{0.0in}}
\item None.
\end{list}
\vspace{2.8ex}

\noindent\textbf{INTERRUPT COMPATIBILITY}
\begin{list}{}{\setlength{\leftmargin}{0.25in}\setlength{\topsep}{0.0in}}
\item This function may be called from non-ISR or ISR software.
\item This function does not protect array accesses in any way, and so
      is not safe to use to operate on data shared with another thread.
\end{list}
\vspace{2.8ex}

\noindent\textbf{EXECUTION TIME}
\begin{list}{}{\setlength{\leftmargin}{0.25in}\setlength{\topsep}{0.0in}}
\item TBD.
\end{list}
\vspace{2.8ex}

\noindent\textbf{FUNCTION NAME MNEMONIC}
\begin{list}{}{\setlength{\leftmargin}{0.25in}\setlength{\topsep}{0.0in}}
\item \emph{U8a}:   operates on an array of UCU\_UINT8.
      \emph{Rs1}:   right shift by one position.
      \emph{U16n}:  the number of elements in the array is specified
                    as a UCU\_UINT16.
\end{list}


%%%%%%%%%%%%%%%%%%%%%%%%%%%%%%%%%%%%%%%%%%%%%%%%%%%%%%%%%%%%%%%%%%%%%%%%%%%%%%%
%%%%%%%%%%%%%%%%%%%%%%%%%%%%%%%%%%%%%%%%%%%%%%%%%%%%%%%%%%%%%%%%%%%%%%%%%%%%%%%
%%%%%%%%%%%%%%%%%%%%%%%%%%%%%%%%%%%%%%%%%%%%%%%%%%%%%%%%%%%%%%%%%%%%%%%%%%%%%%%
\subsection[\emph{UcuAmU16aRs1U16nRxn(\protect\mbox{\protect$\cdot$})}]
           {\emph{UcuAmU16aRs1U16nRxn(\protect\mbox{\protect\boldmath $\cdot$})}}
%Section tag:  rsh1
\label{cami0:srsf0:srsh1}

\index{UcuAmU16aRs1U16nRxn()@\emph{UcuAmU16aRs1U16nRxn($\cdot$)}}%

\noindent\textbf{PROTOTYPE}
\begin {list}{}{\setlength{\leftmargin}{0.25in}\setlength{\topsep}{0.0in}}
\item
\begin{verbatim}
void UcuAmU16aRs1U16nRxn(
                        UCU_UINT16 *in_arr, 
                        UCU_UINT16  in_nelem, 
                        UCU_UINT8   in_pushval
                       )
\end{verbatim}
\end{list}
\vspace{2.8ex}

\noindent\textbf{SYNOPSIS}
\begin{list}{}{\setlength{\leftmargin}{0.25in}\setlength{\topsep}{0.0in}}
\item The \emph{UcuAmU16aRs1U16nRxn($\cdot$)} function is identical in every
      respect to the \emph{UcuAmU8aRs1U16nRxn($\cdot$)} function
      (\S{}\ref{cami0:srsf0:srsh0}, p. \pageref{cami0:srsf0:srsh0})
      except that each element of the array is 16 bits
      rather than 8.
\end{list}
\vspace{2.8ex}


%%%%%%%%%%%%%%%%%%%%%%%%%%%%%%%%%%%%%%%%%%%%%%%%%%%%%%%%%%%%%%%%%%%%%%%%%%%%%%%
%%%%%%%%%%%%%%%%%%%%%%%%%%%%%%%%%%%%%%%%%%%%%%%%%%%%%%%%%%%%%%%%%%%%%%%%%%%%%%%
%%%%%%%%%%%%%%%%%%%%%%%%%%%%%%%%%%%%%%%%%%%%%%%%%%%%%%%%%%%%%%%%%%%%%%%%%%%%%%%
\section{Average Functions}
%Section tag:  afn0
\label{cami0:safn0}


%%%%%%%%%%%%%%%%%%%%%%%%%%%%%%%%%%%%%%%%%%%%%%%%%%%%%%%%%%%%%%%%%%%%%%%%%%%%%%%
%%%%%%%%%%%%%%%%%%%%%%%%%%%%%%%%%%%%%%%%%%%%%%%%%%%%%%%%%%%%%%%%%%%%%%%%%%%%%%%
%%%%%%%%%%%%%%%%%%%%%%%%%%%%%%%%%%%%%%%%%%%%%%%%%%%%%%%%%%%%%%%%%%%%%%%%%%%%%%%
\subsection[\emph{UcuAmU16aAvgFU16nRxn(\protect\mbox{\protect$\cdot$})}]
           {\emph{UcuAmU16aAvgFU16nRxn(\protect\mbox{\protect\boldmath $\cdot$})}}
%Section tag:  avg0
\label{cami0:safn0:savg0}

\index{UcuAmU16aAvgFU16nRxn()@\emph{UcuAmU16aAvgFU16nRxn($\cdot$)}}%

\noindent\textbf{PROTOTYPE}
\begin {list}{}{\setlength{\leftmargin}{0.25in}\setlength{\topsep}{0.0in}}
\item
\begin{verbatim}
UCU_UINT16 UcuAmU16aAvgFU16nRxn(
                               const UCU_UINT16 *a,
                                     UCU_UINT16  n
                               )
\end{verbatim}
\end{list}
\vspace{2.8ex}

\noindent\textbf{SYNOPSIS}
\begin{list}{}{\setlength{\leftmargin}{0.25in}\setlength{\topsep}{0.0in}}
\item Calculates the floor of the average of the array elements,
      $\displaystyle{\left\lfloor{\frac{\sum_{i=0}^{n-1}a_i}{n}}\right\rfloor}$.
\end{list}
\vspace{2.8ex}

\noindent\textbf{INPUTS}
\begin{list}{}{\setlength{\leftmargin}{0.5in}\setlength{\itemindent}{-0.25in}\setlength{\topsep}{0.0in}\setlength{\partopsep}{0.0in}}
\item \emph{\textbf{a}}\\
      Pointer to the first element of the array to be averaged.  If $n=0$, 
      this pointer is not dereferenced.
\item \emph{\textbf{n}}\\
      The total number of elements in the array.  This is frequently
      calculated as \texttt{sizeof(a) / sizeof(a[0])}.
\end{list}
\vspace{2.8ex}

\noindent\textbf{OUTPUTS}
\begin{list}{}{\setlength{\leftmargin}{0.25in}\setlength{\topsep}{0.0in}}
\item The floor of the average of the array elements, 
      $\displaystyle{\left\lfloor{\frac{\sum_{i=0}^{n-1}a_i}{n}}\right\rfloor}$.
\end{list}
\vspace{2.8ex}

\noindent\textbf{EXCEPTION CASES}
\begin{list}{}{\setlength{\leftmargin}{0.25in}\setlength{\topsep}{0.0in}}
\item If $n = 0$, 0 is returned.
\end{list}
\vspace{2.8ex}

\noindent\textbf{INTERRUPT COMPATIBILITY}
\begin{list}{}{\setlength{\leftmargin}{0.25in}\setlength{\topsep}{0.0in}}
\item This function may be called from non-ISR or ISR software.
\item This function does not protect array accesses in any way, and so
      is not safe to use to operate on data written by another thread.
\end{list}
\vspace{2.8ex}

\noindent\textbf{EXECUTION TIME}
\begin{list}{}{\setlength{\leftmargin}{0.25in}\setlength{\topsep}{0.0in}}
\item TBD.
\end{list}
\vspace{2.8ex}

\noindent\textbf{FUNCTION NAME MNEMONIC}
\begin{list}{}{\setlength{\leftmargin}{0.25in}\setlength{\topsep}{0.0in}}
\item \emph{U16a}:   operates on an array of UCU\_UINT16.
      \emph{Avg}:    average.
      \emph{F}:      result is floor'd.
      \emph{U16n}:   the number of elements in the array is specified
                     as a UCU\_UINT16.
\end{list}


%%%%%%%%%%%%%%%%%%%%%%%%%%%%%%%%%%%%%%%%%%%%%%%%%%%%%%%%%%%%%%%%%%%%%%%%%%%%%%%
%%%%%%%%%%%%%%%%%%%%%%%%%%%%%%%%%%%%%%%%%%%%%%%%%%%%%%%%%%%%%%%%%%%%%%%%%%%%%%%
%%%%%%%%%%%%%%%%%%%%%%%%%%%%%%%%%%%%%%%%%%%%%%%%%%%%%%%%%%%%%%%%%%%%%%%%%%%%%%%
\subsection[\emph{UcuAmU16aAvgFDmmU16nRxn(\protect\mbox{\protect$\cdot$})}]
           {\emph{UcuAmU16aAvgFDmmU16nRxn(\protect\mbox{\protect\boldmath $\cdot$})}}
\label{cami0:safn0:savg1}

\index{UcuAmU16aAvgFDmmU16nRxn()@\emph{UcuAmU16aAvgFDmmU16nRxn($\cdot$)}}%

\noindent\textbf{PROTOTYPE}
\begin {list}{}{\setlength{\leftmargin}{0.25in}\setlength{\topsep}{0.0in}}
\item
\begin{verbatim}
UCU_UINT16 UcuAmU16aAvgFDmmU16nRxn(
                                  const UCU_UINT16 *a,
                                        UCU_UINT16  n
                                  )
\end{verbatim}
\end{list}
\vspace{2.8ex}

\noindent\textbf{SYNOPSIS}
\begin{list}{}{\setlength{\leftmargin}{0.25in}\setlength{\topsep}{0.0in}}
\item Calculates the floor of the average of the array elements with the 
      minimum and maximum array elements discarded,
      $\displaystyle{\left\lfloor{\frac{\sum_{i=0}^{n-1}a_i - a_{min} - a_{max}}{n-2}}\right\rfloor}$.
\end{list}
\vspace{2.8ex}

\noindent\textbf{INPUTS}
\begin{list}{}{\setlength{\leftmargin}{0.5in}\setlength{\itemindent}{-0.25in}\setlength{\topsep}{0.0in}\setlength{\partopsep}{0.0in}}
\item \emph{\textbf{a}}\\
      Pointer to the first element of the array to be averaged.  If $n=0$, this
      pointer is not dereferenced.
\item \emph{\textbf{n}}\\
      The total number of elements in the array.  This is frequently
      calculated as \texttt{sizeof(a) / sizeof(a[0])}.
\end{list}
\vspace{2.8ex}

\noindent\textbf{OUTPUTS}
\begin{list}{}{\setlength{\leftmargin}{0.25in}\setlength{\topsep}{0.0in}}
\item The floor of the average of the array elements with the
      minimum and maximum array elements discarded,
      $\displaystyle{\left\lfloor{\frac{\sum_{i=0}^{n-1}a_i - a_{min} - a_{max}}{n-2}}\right\rfloor}$.
\end{list}
\vspace{2.8ex}

\noindent\textbf{EXCEPTION CASES}
\begin{list}{}{\setlength{\leftmargin}{0.25in}\setlength{\topsep}{0.0in}}
\item If $n = 0$, 0 is returned.
\item If $n = 1$, the value of the single array element is returned.
\item If $n = 2$, the floored average of the two array elements,
      $\lfloor (a_1 + a_0)/2 \rfloor$, is returned.
\end{list}
\vspace{2.8ex}

\noindent\textbf{INTERRUPT COMPATIBILITY}
\begin{list}{}{\setlength{\leftmargin}{0.25in}\setlength{\topsep}{0.0in}}
\item This function may be called from non-ISR or ISR software.
\item This function does not protect array accesses in any way, and so
      is not safe to use to operate on data shared with another thread.
\end{list}
\vspace{2.8ex}

\noindent\textbf{EXECUTION TIME}
\begin{list}{}{\setlength{\leftmargin}{0.25in}\setlength{\topsep}{0.0in}}
\item TBD.
\end{list}
\vspace{2.8ex}

\noindent\textbf{FUNCTION NAME MNEMONIC}
\begin{list}{}{\setlength{\leftmargin}{0.25in}\setlength{\topsep}{0.0in}}
\item \emph{U16a}:   operates on an array of UCU\_UINT16.
      \emph{Avg}:    average.
      \emph{F}:      result is floor'd.
      \emph{Dmm}:    \emph{d}iscard \emph{m}inimum and \emph{m}aximum elements.
      \emph{U16n}:   the number of elements in the array is specified
                     as a UCU\_UINT16.
\end{list}

%%%%%%%%%%%%%%%%%%%%%%%%%%%%%%%%%%%%%%%%%%%%%%%%%%%%%%%%%%%%%%%%%%%%%%%%%%%%%%%
%%%%%%%%%%%%%%%%%%%%%%%%%%%%%%%%%%%%%%%%%%%%%%%%%%%%%%%%%%%%%%%%%%%%%%%%%%%%%%%
%%%%%%%%%%%%%%%%%%%%%%%%%%%%%%%%%%%%%%%%%%%%%%%%%%%%%%%%%%%%%%%%%%%%%%%%%%%%%%%
\subsection[\emph{UcuAmU16aAvgFDmmRmmU16nRxn(\protect\mbox{\protect$\cdot$})}]
           {\emph{UcuAmU16aAvgFDmmRmmU16nRxn(\protect\mbox{\protect\boldmath $\cdot$})}}
\label{cami0:safn0:savg2}

\index{UcuAmU16aAvgFDmmRmmU16nRxn()@\emph{UcuAmU16aAvgFDmmRmmU16nRxn($\cdot$)}}%

\noindent\textbf{PROTOTYPE}
\begin {list}{}{\setlength{\leftmargin}{0.25in}\setlength{\topsep}{0.0in}}
\item
\begin{verbatim}
void UcuAmU16aAvgFDmmRmmU16nRxn(
                               const UCU_UINT16 *a, 
                                     UCU_UINT16  n, 
                                     UCU_UINT16 *avg, 
                                     UCU_UINT16 *a_min, 
                                     UCU_UINT16 *a_max
                               )
\end{verbatim}
\end{list}
\vspace{2.8ex}

\noindent\textbf{SYNOPSIS}
\begin{list}{}{\setlength{\leftmargin}{0.25in}\setlength{\topsep}{0.0in}}
\item Calculates the floor of the average of the array elements with the 
      minimum and maximum array elements discarded,
      $\displaystyle{\left\lfloor{\frac{\sum_{i=0}^{n-1}a_i - a_{min} - a_{max}}{n-2}}\right\rfloor}$.
      The values of the minimum and maximum elements are also returned.
\end{list}
\vspace{2.8ex}

\noindent\textbf{INPUTS}
\begin{list}{}{\setlength{\leftmargin}{0.5in}\setlength{\itemindent}{-0.25in}\setlength{\topsep}{0.0in}\setlength{\partopsep}{0.0in}}
\item \emph{\textbf{a}}\\
      Pointer to the first element of the array to be averaged.  If $n=0$, this
      pointer is not dereferenced.
\item \emph{\textbf{n}}\\
      The total number of elements in the array.  This is frequently
      calculated as \texttt{sizeof(a) / sizeof(a[0])}.
\end{list}
\vspace{2.8ex}

\noindent\textbf{OUTPUTS}
\begin{list}{}{\setlength{\leftmargin}{0.5in}\setlength{\itemindent}{-0.25in}\setlength{\topsep}{0.0in}\setlength{\partopsep}{0.0in}}
\item \emph{\textbf{*avg}}\\
      The average of the array elements with the minimum and maximum
      elements removed:
      $\displaystyle{\left\lfloor{\frac{\sum_{i=0}^{n-1}a_i - a_{min} - a_{max}}{n-2}}\right\rfloor}$.
\item \emph{\textbf{*a\_min}}\\
      The value of the minimum element of the array that was removed from
      the average via subtraction.  All cases with duplicate elements (for example, an
      array with every element identical) are handled correctly so that this value represents the minimum
      of the elements in the array.
\item \emph{\textbf{*a\_max}}\\
      The value of the maximum element of the array that was removed from
      the average via subtraction.  All cases with duplicate elements (for example, an
      array with every element identical) are handled correctly so that this value represents the maximum
      of the elements in the array.
\end{list}
\vspace{2.8ex}

\noindent\textbf{EXCEPTION CASES}
\begin{list}{}{\setlength{\leftmargin}{0.25in}\setlength{\topsep}{0.0in}}
\item If $n = 0$, 0 is returned for the average, the minimum element, and the maximum element.
\item If $n = 1$, the value of the single array element is returned for the average, the minimum element, and
      the maximum element.
\item If $n = 2$, the floored average of the two array elements,
      $\lfloor (a_1 + a_0)/2 \rfloor$, is returned as the average.  The smaller element is
      returned as the minimum element, and the larger element is returned as the maximum element.
\end{list}
\vspace{2.8ex}

\noindent\textbf{INTERRUPT COMPATIBILITY}
\begin{list}{}{\setlength{\leftmargin}{0.25in}\setlength{\topsep}{0.0in}}
\item This function may be called from non-ISR or ISR software.
\item This function does not protect array accesses in any way, and so
      is not safe to use to operate on data shared with another thread.
\end{list}
\vspace{2.8ex}

\noindent\textbf{EXECUTION TIME}
\begin{list}{}{\setlength{\leftmargin}{0.25in}\setlength{\topsep}{0.0in}}
\item TBD.
\end{list}
\vspace{2.8ex}

\noindent\textbf{FUNCTION NAME MNEMONIC}
\begin{list}{}{\setlength{\leftmargin}{0.25in}\setlength{\topsep}{0.0in}}
\item \emph{U16a}:   operates on an array of UCU\_UINT16.
      \emph{Avg}:    average.
      \emph{F}:      result is floor'd.
      \emph{Dmm}:    \emph{d}iscard \emph{m}inimum and \emph{m}aximum elements.
      \emph{Rmm}:    \emph{r}eturn the \emph{m}inimum and \emph{m}aximum elements
                     (that were discarded).
      \emph{U16n}:   the number of elements in the array is specified
                     as a UCU\_UINT16.
\end{list}


%%%%%%%%%%%%%%%%%%%%%%%%%%%%%%%%%%%%%%%%%%%%%%%%%%%%%%%%%%%%%%%%%%%%%%%%%%
\noindent\begin{figure}[!b]
\noindent\rule[-0.25in]{\textwidth}{1pt}
\begin{tiny}
\begin{verbatim}
$RCSfile: c_ami0.tex,v $
$Source: /home/dashley/cvsrep/uculib01/uculib01/doc/manual/c_ami0/c_ami0.tex,v $
$Revision: 1.12 $
$Author: dashley $
$Date: 2010/02/24 14:50:36 $
\end{verbatim}
\end{tiny}
\noindent\rule[0.25in]{\textwidth}{1pt}
\end{figure}

%%%%%%%%%%%%%%%%%%%%%%%%%%%%%%%%%%%%%%%%%%%%%%%%%%%%%%%%%%%%%%%%%%%%%%%%%%%%%%%
%$Log: c_ami0.tex,v $
%Revision 1.12  2010/02/24 14:50:36  dashley
%Function documentation enhanced.
%
%Revision 1.11  2010/02/23 18:51:06  dashley
%Edits.
%
%Revision 1.10  2010/02/23 15:51:49  dashley
%Edits.
%
%Revision 1.9  2010/02/17 15:04:46  dashley
%Function documentation enhanced.
%
%Revision 1.8  2010/01/29 20:37:46  dashley
%Functions added.
%
%Revision 1.7  2010/01/29 06:59:20  dashley
%Addition of UcuAmU16aAvgFU16nRxn() function.
%
%Revision 1.6  2010/01/29 03:06:11  dashley
%Addition of UcuAmU16aRs1U16nRxn() function.
%
%Revision 1.5  2010/01/28 22:46:47  dashley
%Addition of UcuAmU8aRs1U16nRxn() function.
%
%Revision 1.4  2010/01/28 21:18:32  dashley
%a)Chapter start quotes removed.
%b)Aesthetic comment line added at the bottom of most files.
%
%Revision 1.3  2010/01/28 21:00:29  dashley
%Functions added.
%
%Revision 1.2  2010/01/28 19:13:55  dashley
%Integration of UcuAmU8aDnbzU16nRxn().
%
%Revision 1.1  2010/01/24 05:11:56  dashley
%Initial checkin.
%%%%%%%%%%%%%%%%%%%%%%%%%%%%%%%%%%%%%%%%%%%%%%%%%%%%%%%%%%%%%%%%%%%%%%%%%%%%%%%

