%$Header: /cvsroot/esrg/sfesrg/esrgpcpj/hyreach/doc/hyreachm/s_iff0/s_iff0.tex,v 1.4 2002/01/29 17:04:00 dtashley Exp $
%
\section{Input File Format}
\label{siff0}

\swname{} accepts a model input file which specifies the model 
to be verified for reachability properties.  This section of the
document specifies the format of this file.


%%%%%%%%%%%%%%%%%%%%%%%%%%%%%%%%%%%%%%%%%%%%%%%%%%%%%%%%%%%%%%%%%%%%%%%%%%%%%%%
%%%%%%%%%%%%%%%%%%%%%%%%%%%%%%%%%%%%%%%%%%%%%%%%%%%%%%%%%%%%%%%%%%%%%%%%%%%%%%%
%%%%%%%%%%%%%%%%%%%%%%%%%%%%%%%%%%%%%%%%%%%%%%%%%%%%%%%%%%%%%%%%%%%%%%%%%%%%%%%
\subsection{Overview And Essential Characteristics Of The Model Input File Format}
%Subsection Tag: OEC0
\label{siff0:soec0}

The model input file is an ASCII text file with the following characteristics.

\begin{enumerate}
\item The file is a plain ASCII text file rather than a binary file.
      This means the file can be prepared using an ordinary text editor, a 
      \texttt{.BAT} file, a script, etc.
\item The file is token- rather than record-oriented.  
      This means the file has a relatively free aesthetic format and can be 
      formatted to suit the tastes of the preparer.
\item Keywords and variables are case-sensitive.  
      \emph{Catsanddogs} is different than \emph{CatsAndDogs}.
\end{enumerate}

The remainder of the this section (Section \ref{siff0})
describes the format, and finally concludes with an example.


%%%%%%%%%%%%%%%%%%%%%%%%%%%%%%%%%%%%%%%%%%%%%%%%%%%%%%%%%%%%%%%%%%%%%%%%%%%%%%%
%%%%%%%%%%%%%%%%%%%%%%%%%%%%%%%%%%%%%%%%%%%%%%%%%%%%%%%%%%%%%%%%%%%%%%%%%%%%%%%
%%%%%%%%%%%%%%%%%%%%%%%%%%%%%%%%%%%%%%%%%%%%%%%%%%%%%%%%%%%%%%%%%%%%%%%%%%%%%%%
\subsection{Allowable Character Set}
%Subsection Tag: ACS0
\label{siff0:sacs0}

The input file is processed as an ASCII text file (\textbf{not} UNICODE!).
All characters (bytes) in the input file must have a binary value in the range
of 1-127.  Any character outside of that range will cause a fatal error.

Note that all of the traditional characters (letters, numbers, punctuation, 
CR, LF, etc.) fall within the range.  0 is prohibited simply because the software
uses it internally as an end-of-string marker.


%%%%%%%%%%%%%%%%%%%%%%%%%%%%%%%%%%%%%%%%%%%%%%%%%%%%%%%%%%%%%%%%%%%%%%%%%%%%%%%
%%%%%%%%%%%%%%%%%%%%%%%%%%%%%%%%%%%%%%%%%%%%%%%%%%%%%%%%%%%%%%%%%%%%%%%%%%%%%%%
%%%%%%%%%%%%%%%%%%%%%%%%%%%%%%%%%%%%%%%%%%%%%%%%%%%%%%%%%%%%%%%%%%%%%%%%%%%%%%%
\subsection{Pre-Tokenization Processing}
%Subsection Tag: PTP0
\label{siff0:sptp0}

\swname{} breaks the input file into an ordered list of tokens.  These tokens
drive an automaton which processes them.  Chaos (or, more specifically, error
messages) result when the ordered list of tokens cannot properly excite
the automaton.

Some processing is performed before the ordered list of tokens is created.  This
section describes all such processing.


%%%%%%%%%%%%%%%%%%%%%%%%%%%%%%%%%%%%%%%%%%%%%%%%%%%%%%%%%%%%%%%%%%%%%%%%%%%%%%%
%%%%%%%%%%%%%%%%%%%%%%%%%%%%%%%%%%%%%%%%%%%%%%%%%%%%%%%%%%%%%%%%%%%%%%%%%%%%%%%
%%%%%%%%%%%%%%%%%%%%%%%%%%%%%%%%%%%%%%%%%%%%%%%%%%%%%%%%%%%%%%%%%%%%%%%%%%%%%%%
\subsubsection{Format And Removal Of Comments}
%Subsubsection Tag: FRC0
\label{siff0:sptp0:sfrc0}

\swname{} honors C++-style comments.  \texttt{/*} and \texttt{*/} are
accepted to begin and end comments, and \texttt{//} is accepted to
create a comment which extends to the end of the line.

Comments and string literals are processed in the same pass.
This means that string literals which begin within or exist entirely within
comments are not detected.  Similarly, string literals which occur within a comment
are not detected, and one style of comment that occurs within the other style of
comment is not detected.


%%%%%%%%%%%%%%%%%%%%%%%%%%%%%%%%%%%%%%%%%%%%%%%%%%%%%%%%%%%%%%%%%%%%%%%%%%%%%%%
%%%%%%%%%%%%%%%%%%%%%%%%%%%%%%%%%%%%%%%%%%%%%%%%%%%%%%%%%%%%%%%%%%%%%%%%%%%%%%%
%%%%%%%%%%%%%%%%%%%%%%%%%%%%%%%%%%%%%%%%%%%%%%%%%%%%%%%%%%%%%%%%%%%%%%%%%%%%%%%
\subsubsection{Format And Processing Of String Literals}
%Subsubsection Tag: FRS0
\label{siff0:sptp0:sfrs0}

As described in Section \ref{siff0:sptk0}, \swname{} forms tokens based
exclusively on separating whitespace.  This means that without some
escape mechanism, it would be impossible to create tokens which contain 
whitespace.

String literals are text which is enclosed in double quotes (\texttt{"}).
For example, \texttt{"Now is the time."} is a string literal.

String literals are processed in the same pass as comments.  This means that
comments enclosed within a string literal will not be detected (see also the remarks
in Section \ref{siff0:sptp0:sfrc0}).

String literals may not span lines.

Within a string literal, \texttt{$\backslash$"} may be used to include the double quote
character as part of the literal.  This is the same escape mechanism
traditionally used in programming languages.  No other escape sequences
are honored.

Within a string literal, any tab character is converted to a single space.  The
reason this is done is that string literals are at this time only used to
contain descriptions.  Any tab characters may create undesirable effects in the
program output.

Generally, \swname{} may format descriptions by breaking them at whitespace in
order to fit them to the line width and indenting available.  This behavior
may not be suppressed.

Any number of successive string literals may be combined by using the special operator
\texttt{+"} rather than \texttt{"} to start a string literal.  
The operator \texttt{+"} causes the string literal currently being defined to
be concatenated to the previous token.  The \texttt{+"} operator may be used
to create long descriptions which span several lines.  Note that spaces are
not automatically inserted, so these must be included.

Note that the notion of the string literal is a parsing convenienience to allow
the specification of tokens which contain spaces and which are longer than
a comfortable single line of text.  The double-quote characters (\texttt{"})
and special concatenation operators (\texttt{+"}) are discarded and not 
included in the internal data representation.

A token which contains no spaces can be enclosed in double-quotes with no 
effect, although there is no practical reason to ever do this.


%%%%%%%%%%%%%%%%%%%%%%%%%%%%%%%%%%%%%%%%%%%%%%%%%%%%%%%%%%%%%%%%%%%%%%%%%%%%%%%
%%%%%%%%%%%%%%%%%%%%%%%%%%%%%%%%%%%%%%%%%%%%%%%%%%%%%%%%%%%%%%%%%%%%%%%%%%%%%%%
%%%%%%%%%%%%%%%%%%%%%%%%%%%%%%%%%%%%%%%%%%%%%%%%%%%%%%%%%%%%%%%%%%%%%%%%%%%%%%%
\subsection{Formation Of Tokens}
%Subsection Tag: PTK0
\label{siff0:sptk0}

\swname{} breaks the input file into an ordered list of tokens.  These tokens
drive an automaton which processes them.  Chaos (or, more specifically, error
messages) result when the ordered list of tokens cannot properly drive
the automaton.

String literals (Section \ref{siff0:sptp0:sfrs0}) are necessary only 
when a token contains spaces
or is longer than comfortably fits on a line of text.

\swname{} will never identify a potential token based on a criterium other than
surrounding whitespace.  For example, the classic example of token 
recognition is the way that a C-language lexical analyzer would 
parse a statement such as \texttt{c+++=3;}.  \swname{}'s lexical analysis strategy
is simpler and would require that such a statement be written 
\texttt{c ++ += 3 ;} (in other words, it is required that the user in
some sense do the lexical analysis).

We decline to specify the syntax of the \swname{} model input file
\emph{precisely}.  In practice there should be no scenarios where a model file 
can be interpreted ambiguously, and nearly no scenarios where any error messages
are not straightforward to diagnose.


%%%%%%%%%%%%%%%%%%%%%%%%%%%%%%%%%%%%%%%%%%%%%%%%%%%%%%%%%%%%%%%%%%%%%%%%%%
\noindent\begin{figure}[!b]
\noindent\rule[-0.25in]{\textwidth}{1pt}
\begin{tiny}
\begin{verbatim}
$RCSfile: s_iff0.tex,v $
$Source: /cvsroot/esrg/sfesrg/esrgpcpj/hyreach/doc/hyreachm/s_iff0/s_iff0.tex,v $
$Revision: 1.4 $
$Author: dtashley $
$Date: 2002/01/29 17:04:00 $
\end{verbatim}
\end{tiny}
\noindent\rule[0.25in]{\textwidth}{1pt}
\end{figure}
%%%%%%%%%%%%%%%%%%%%%%%%%%%%%%%%%%%%%%%%%%%%%%%%%%%%%%%%%%%%%%%%%%%%%%%%%%
%$Log: s_iff0.tex,v $
%Revision 1.4  2002/01/29 17:04:00  dtashley
%Version control info added, and minor edits.
%
%Revision 1.3  2001/12/01 21:29:54  dtashley
%Safety checkin after edits.
%
%Revision 1.2  2001/09/26 04:51:13  dtashley
%Edits.
%
%Revision 1.1  2001/09/26 02:31:14  dtashley
%Initial checkin, and some edits of main TEX file.
%
%End of S_IFF0.TEX.
