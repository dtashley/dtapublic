%$Header: /cvsroot/esrg/sfesrg/esrgpcpj/doc/engman01/c_rsi0/c_rsi0.tex,v 1.2 2002/07/29 16:53:09 dtashley Exp $
%
\chapter{Required Software Installations}
\label{crsi0}

The build process for Windows executables, Linux executables, 
Windows InstallShield Express installation executables, 
and \LaTeX{} documents requires many pieces of software.
Some of this software is free, and some is commercial
(and quite expensive).  In this chapter we describe 
all software required \emph{The \tsname{}} and all supporting
documentation on both Windows and *Nix platforms.  Note that
depending on what is being built, not all software may
be required.


%%%%%%%%%%%%%%%%%%%%%%%%%%%%%%%%%%%%%%%%%%%%%%%%%%%%%%%%%%%%%%%%%%%%%%%%%%
%%%%%%%%%%%%%%%%%%%%%%%%%%%%%%%%%%%%%%%%%%%%%%%%%%%%%%%%%%%%%%%%%%%%%%%%%%
%%%%%%%%%%%%%%%%%%%%%%%%%%%%%%%%%%%%%%%%%%%%%%%%%%%%%%%%%%%%%%%%%%%%%%%%%%
\section{Windows Software}
\label{crsi0:swso0}


%%%%%%%%%%%%%%%%%%%%%%%%%%%%%%%%%%%%%%%%%%%%%%%%%%%%%%%%%%%%%%%%%%%%%%%%%%
%%%%%%%%%%%%%%%%%%%%%%%%%%%%%%%%%%%%%%%%%%%%%%%%%%%%%%%%%%%%%%%%%%%%%%%%%%
%%%%%%%%%%%%%%%%%%%%%%%%%%%%%%%%%%%%%%%%%%%%%%%%%%%%%%%%%%%%%%%%%%%%%%%%%%
\subsection{Adobe Acrobat Version 4.0}
\label{crsi0:swso0:saaf0}

\index{Adobe!Acrobat}\index{Adobe!Acrobat Distiller}
\emph{Adobe Acrobat}, which costs abour \$250, contains a program
called \emph{Acrobat Distiller}.
Acrobat Distiller 
is the best application
we've found for 
creating \index{.PDF file@\texttt{.PDF} file}\texttt{.PDF} files from
\LaTeX{} source code.  The most successful
conversion approach seems to be 
to create an output file for a PostScript printer,
then to use Acrobat Distiller to convert it to 
\texttt{.PDF}.

To the best of our knowledge, it is not possible to purchase
Acrobat Distiller separate from Adobe Acrobat---Adobe
Acrobat must be purchased in order to obtain Acrobat Distiller.

When Adobe Acrobat is installed, all of the installation defaults
should be accepted.  \index{Tcl/Tk@\emph{Tcl/Tk}}Tcl/Tk 
scripts and \index{.BAT file@\texttt{.BAT} file} \texttt{.BAT}
files used in building \emph{The \tsname{}} rely on the 
executables being in the default locations.


%%%%%%%%%%%%%%%%%%%%%%%%%%%%%%%%%%%%%%%%%%%%%%%%%%%%%%%%%%%%%%%%%%%%%%%%%%
%%%%%%%%%%%%%%%%%%%%%%%%%%%%%%%%%%%%%%%%%%%%%%%%%%%%%%%%%%%%%%%%%%%%%%%%%%
%%%%%%%%%%%%%%%%%%%%%%%%%%%%%%%%%%%%%%%%%%%%%%%%%%%%%%%%%%%%%%%%%%%%%%%%%%
\subsection{\tsname{} Version \tsversion{}}
\label{crsi0:swso0:sets0}

%Note:  must hard code in the index entry below:  reason is processing
%       order of MAKEINDEX and LATEX.  As it is below, a global name 
%       change will get correct 

\index{ESRG Tool Set@\emph{\tsname{}}}
The \index{Tcl/Tk@\emph{Tcl/Tk}}Tcl/Tk interpreter 
(\index{EsrgConsole@\emph{EsrgConsole}}\emph{EsrgConsole}, 
the statically linked version
of \index{Wish@\emph{Wish}}\emph{Wish}, with extensions)
from \emph{The \tsname{}} is necessary to build
\emph{The \tsname{}} on Windows platforms.  Thus, \emph{The \tsname{}}
is necessary to build \emph{The \tsname{}}.  

At the present time, EsrgConsole is used only in the production
of the companion book.  However, in future revisions of
the tool set, the build complexity may become great enough that
it is used in the production of the Windows executables.

When installing \emph{The \tsname{}}, it is not necessary to
install the entire tool set---only the statically linked
Tcl/Tk interpreter (EsrgConsole) is required.  Copying the executable
from one computer to another will usually be adequate.  It is also not
necessary that the executable go in a certain place, since starting
the script interpreter is always a manual step in the build process.
No specific version of EsrgConsole relative to the version of the
tool set being built is \emph{actually} required; but
because it is not a good idea to allow freedom in the build process,
we canonically require that the version of EsrgConsole used for builds
should be the last stable release before the version being built.


%%%%%%%%%%%%%%%%%%%%%%%%%%%%%%%%%%%%%%%%%%%%%%%%%%%%%%%%%%%%%%%%%%%%%%%%%%
%%%%%%%%%%%%%%%%%%%%%%%%%%%%%%%%%%%%%%%%%%%%%%%%%%%%%%%%%%%%%%%%%%%%%%%%%%
%%%%%%%%%%%%%%%%%%%%%%%%%%%%%%%%%%%%%%%%%%%%%%%%%%%%%%%%%%%%%%%%%%%%%%%%%%
\subsection{InstallShield Express Version 3.03}
\label{crsi0:swso0:sise0}

\index{InstallShield Express@\emph{InstallShield Express}}
\emph{InstallShield Express}, which costs about
\$200, is the program used to create
the installation executable for Windows platforms.

Version 3.00 of the program contained some serious bugs, which were
corrected in subsequent maintenance releases.  At the time of this
writing, Version 3.03 is the most modern maintenance release.

When installing InstallShield Express, if the installation media
(typically a CD) are for a version earlier than version 3.03, it is
recommended to download maintenance release 3.03 from 
Installshield's web site and to install the maintenance release
without first installing from the installation media (note that the
maintenance release will require the serial number from the installation
media).  The reason for this recommendation is that the maintenance release
is standalone and just writes over earlier versions, anyway.
When installing InstallShield Express, it is not required that
installation defaults be accepted.  The generation of
Windows installation executables is at this time a manual process,
and no \index{Tcl/Tk@\emph{Tcl/Tk}}Tcl/Tk 
scripts or \index{.BAT file@\texttt{.BAT} file} \texttt{.BAT}
files rely on the location of the InstallShield Express executables.


%%%%%%%%%%%%%%%%%%%%%%%%%%%%%%%%%%%%%%%%%%%%%%%%%%%%%%%%%%%%%%%%%%%%%%%%%%
%%%%%%%%%%%%%%%%%%%%%%%%%%%%%%%%%%%%%%%%%%%%%%%%%%%%%%%%%%%%%%%%%%%%%%%%%%
%%%%%%%%%%%%%%%%%%%%%%%%%%%%%%%%%%%%%%%%%%%%%%%%%%%%%%%%%%%%%%%%%%%%%%%%%%
\section{*Nix Software}
\label{crsi0:snix0}





%%%%%%%%%%%%%%%%%%%%%%%%%%%%%%%%%%%%%%%%%%%%%%%%%%%%%%%%%%%%%%%%%%%%%%%%%%
%%%%%%%%%%%%%%%%%%%%%%%%%%%%%%%%%%%%%%%%%%%%%%%%%%%%%%%%%%%%%%%%%%%%%%%%%%
%%%%%%%%%%%%%%%%%%%%%%%%%%%%%%%%%%%%%%%%%%%%%%%%%%%%%%%%%%%%%%%%%%%%%%%%%%
\noindent\begin{figure}[!b]
\noindent\rule[-0.25in]{\textwidth}{1pt}
\begin{tiny}
\begin{verbatim}
$RCSfile: c_rsi0.tex,v $
$Source: /cvsroot/esrg/sfesrg/esrgpcpj/doc/engman01/c_rsi0/c_rsi0.tex,v $
$Revision: 1.2 $
$Author: dtashley $
$Date: 2002/07/29 16:53:09 $
\end{verbatim}
\end{tiny}
\noindent\rule[0.25in]{\textwidth}{1pt}
\end{figure}
%%%%%%%%%%%%%%%%%%%%%%%%%%%%%%%%%%%%%%%%%%%%%%%%%%%%%%%%%%%%%%%%%%%%%%%%%%
%$Log: c_rsi0.tex,v $
%Revision 1.2  2002/07/29 16:53:09  dtashley
%Safety checkins before being moved to WSU server Kalman.
%
%Revision 1.1  2002/06/25 02:24:21  dtashley
%Initial checkin.
%
%End of C_RSI0.TEX.
