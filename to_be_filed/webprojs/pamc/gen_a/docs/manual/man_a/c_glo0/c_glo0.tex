%$Header: /home/dashley/cvsrep/e3ft_gpl01/e3ft_gpl01/webprojs/pamc/gen_a/docs/manual/man_a/c_glo0/c_glo0.tex,v 1.2 2007/06/04 00:26:38 dashley Exp $

\chapter{Glossary Of Terms}
\markboth{GLOSSARY OF TERMS}{GLOSSARY OF TERMS}

\label{cglo0}

\begin{vworktermglossaryenum}


\item \textbf{cardinality}\index{cardinality}

      The cardinality of a set is the
      number of elements in the set.  In this work, the cardinality
      of a set is denoted $n()$.  For example, 
      $n(\{12,29,327\}) = 3$.

\item \textbf{coprime}\index{coprime}

      Two integers that share no prime factors are \emph{coprime}.
      \emph{Example:}
      6 and 7 are coprime, whereas 6 and 8 are not.

\item \textbf{GMP}\index{GMP}

      The \emph{G}NU \emph{M}ultiple \emph{P}recision library.
      The GMP is an arbitrary-precision integer, rational number,
      and floating-point library that places no restrictions on
      size of integers or number of significant digits in floating-point
      numbers.  This 
      library is famous because it is the fastest of its
      kind, and generally uses asymptotically superior algorithms.

\item \textbf{greatest common divisor (g.c.d.)}

      The greatest common divisor of two integers is the largest
      integer which divides both integers without a remainder.
      \emph{Example:} the g.c.d. of 30 and 42 is 6.

\item \textbf{irreducible}

      A rational number $p/q$ where $p$ and $q$ are coprime
      is said to be \emph{irreducible}.
      Equivalently, it may be stated that $p$ and $q$ share no prime factors
      or that the greatest common divisor of
      $p$ and $q$ is 1.

\item \textbf{KPH}

      Kilometers per hour.

\item \textbf{limb}\index{limb}

      An integer of a size which a machine can manipulate natively
      that is arranged in an array to create a larger
      integer which the machine cannot manipulate natively and must be
      manipulated through arithmetic subroutines.

\item \textbf{limbsize}\index{limbsize}

      The size, in bits, of a limb.  The limbsize usually represents
      the size of integer that a machine can manipulate directly
      through machine instructions.  For an inexpensive microcontroller,
      8 or 16 is a typical limbsize.  For a personal computer or 
      workstation, 32 or 64 is a typical limbsize.

\item \textbf{MPH}

      Miles per hour.

\end{vworktermglossaryenum}

%%%%%%%%%%%%%%%%%%%%%%%%%%%%%%%%%%%%%%%%%%%%%%%%%%%%%%%%%%%%%%%%%%%%%%%%%%

\noindent\begin{figure}[!b]
\noindent\rule[-0.25in]{\textwidth}{1pt}
\begin{tiny}
\begin{verbatim}
$RCSfile: c_glo0.tex,v $
$Source: /home/dashley/cvsrep/e3ft_gpl01/e3ft_gpl01/webprojs/pamc/gen_a/docs/manual/man_a/c_glo0/c_glo0.tex,v $
$Revision: 1.2 $
$Author: dashley $
$Date: 2007/06/04 00:26:38 $
\end{verbatim}
\end{tiny}
\noindent\rule[0.25in]{\textwidth}{1pt}
\end{figure}

%%%%%%%%%%%%%%%%%%%%%%%%%%%%%%%%%%%%%%%%%%%%%%%%%%%%%%%%%%%%%%%%%%%%%%%%%%%%%%%
%$Log: c_glo0.tex,v $
%Revision 1.2  2007/06/04 00:26:38  dashley
%Edits.
%
%Revision 1.1  2007/06/03 23:36:13  dashley
%Initial checkin.
%
%End of file C_GLO0.TEX
